
% resumo em português
\begin{resumo}[Resumo] \noindent 
Este trabalho apresenta uma pesquisa que tem como objetivo compreender, por meio da elaboração de uma linguagem específica de domínio, a viabilidade de melhoria na comunicação entre usuários de negócio e desenvolvedores, visando o aumento na produtividade da especificação de requisitos e da implementação de regras concernentes ao sistema de cotas da rede de ensino pública federal. A classificação de candidatos cotistas é garantida por meio da lei Nº 12.711/2012, em conjunto com os decretos Nº 7.824 e Nº 9.034, os quais passaram por mudanças em suas diretrizes nos anos de 2012, 2016 e 2017. Adicionalmente a essas mudanças, podem ocorrer situações em que a interpretação de lei é alterada. Essas dificuldades podem resultar em atrasos na aderência à legislação, assim como falhas na comunicação entre desenvolvedores e especialistas de negócio. Portanto, a criação de uma linguagem que expresse regras de distribuição de vagas de maneira mais clara poderá auxiliar na geração de código de classificação com menor dependência de conhecimento técnico em programação, o que pode contribuir para a celeridade dos processos de ingresso em futuras alterações de regras de classificação. Como metodologia realizou-se uma pesquisa de natureza qualitativa com 20 usuários, de modo a verificar as dificuldades de compreensão e uso da linguagem desenvolvida na ferramenta Meta Programming System (MPS) da JetBrains, além do desenvolvimento de uma \gls{API} como prova de conceito para a respectiva geração do serviço de classificação. Para tanto, esse trabalho foi baseado em um levantamento do histórico de mudanças no sistema de controle de versões do \gls{IFSC}, no qual foram identificadas as principais alterações realizadas em função de versões de lei até o momento. Como resultado da presente pesquisa obteve-se o comparativo de dificuldades e preocupações sobre o uso da DSL entre 4 (quatro) grupos de usuários com diferentes características de formação e experiência. Após essa análise foram implementadas algumas melhorias na linguagem, com relação aos comentários e sugestões dos usuários. Ademais, com os testes da \gls{API}, foi possível comparar os resultados da implementação da linguagem com os dados históricos de processamento de candidatos do IFSC em 16 processos seletivos. Por fim, conclui-se que a aplicação da DSL Cotas pode contribuir para a melhoria de comunicação e de entendimento entre os diferentes perfis de conhecimento dos envolvidos, além de propor um novo meio de aplicação prática-teoria de linguagens de domínio específicas. Ressalta-se a relevância social desse estudo, no que diz respeito às ações institucionais para atendimento de demandas que tratam de questões sobre a inclusão social como prioridade nos processos seletivos de ingresso.

% \noindent %- o resumo deve ter apenas 1 parágrafo e sem recuo de texto na primeira linha, essa tag remove o recuo. Não pode haver quebra de linha.

 \vspace{\onelineskip}
    
 \noindent
 \textbf{Palavras-chaves}: Linguagem de Domínio Específico. Meta Programming System. Lei de Cotas 12.711/2012. Regras para Classificação de Candidatos. Rede Federal de Ensino.
\end{resumo}



% resumo em inglês
\begin{resumo}[Abstract]
\begin{otherlanguage*}{english}

 \noindent
\lipsum[5]

   \vspace{\onelineskip}
 
   \noindent 
   \textbf{Key-words}: Domain Specific Language. Meta Programming System. Vacancy Reservation Law 12.711/2012.
 \end{otherlanguage*}
 \end{resumo}
