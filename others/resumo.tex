
% resumo em português
\begin{resumo}[Resumo] \noindent 
Este trabalho apresenta uma pesquisa que tem como objetivo compreender, por meio da elaboração de uma linguagem específica de domínio, a viabilidade de melhoria na comunicação entre usuários de negócio e desenvolvedores, visando o aumento na produtividade da especificação de requisitos e da implementação de regras concernentes ao sistema de cotas da rede de ensino pública federal. A classificação de candidatos cotistas é garantida por meio da lei nº 12.711/2012, em conjunto com os decretos nº 7.824 e nº 9.034, os quais passaram por mudanças em suas diretrizes nos anos de 2012, 2016 e 2017. Adicionalmente a essas mudanças, podem ocorrer situações em que a interpretação de lei é alterada. Essas dificuldades podem resultar em atrasos na aderência à legislação, assim como falhas na comunicação entre desenvolvedores e especialistas de negócio. Portanto, a criação de uma linguagem que expresse regras de distribuição de vagas de maneira mais clara poderá auxiliar na geração de código de classificação com menor dependência de conhecimento técnico em programação, o que pode contribuir para a celeridade dos processos de ingresso em futuras alterações de regras de classificação. Como metodologia realizou-se uma pesquisa de natureza qualitativa com 20 usuários, de modo a verificar as dificuldades de compreensão e uso da linguagem desenvolvida na ferramenta Meta Programming System (MPS) da JetBrains, além do desenvolvimento de uma \gls{API} como prova de conceito para a respectiva geração do serviço de classificação. Para tanto, esse trabalho foi baseado em um levantamento do histórico de mudanças no sistema de controle de versões do \gls{IFSC}, no qual foram identificadas as principais alterações realizadas em função de versões de lei até o momento. Como resultado da presente pesquisa obteve-se o comparativo de dificuldades e preocupações sobre o uso da DSL entre 4 (quatro) grupos de usuários com diferentes características de formação e experiência. Após essa análise foram implementadas algumas melhorias na linguagem, com relação aos comentários e sugestões dos usuários. Ademais, com os testes da \gls{API}, foi possível comparar os resultados da implementação da linguagem com os dados históricos de processamento de candidatos do IFSC em 16 processos seletivos. Por fim, conclui-se que a aplicação da DSL Cotas pode contribuir para a melhoria de comunicação e de entendimento entre os diferentes perfis de conhecimento dos envolvidos, além de propor um novo meio de aplicação prática-teoria de linguagens de domínio específicas. Ressalta-se a relevância social desse estudo, no que diz respeito às ações institucionais para atendimento de demandas que tratam de questões sobre a inclusão social como prioridade nos processos seletivos de ingresso.

% \noindent %- o resumo deve ter apenas 1 parágrafo e sem recuo de texto na primeira linha, essa tag remove o recuo. Não pode haver quebra de linha.

 \vspace{\onelineskip}
    
 \noindent
 \textbf{Palavras-chaves}: Linguagem de Domínio Específico. Meta Programming System. Lei de Cotas 12.711/2012. Regras para Classificação de Candidatos. Rede Federal de Ensino.
\end{resumo}



% resumo em inglês
\begin{resumo}[Abstract]
\begin{otherlanguage*}{english}

 \noindent
This work presents a research that aims to understand the feasibility of improving communication between business users and developers, through the elaboration of a domain specific language, intending to increase the productivity during the requirements specification and at the implementation of vacancy reservation law rules, required for federal public schools. The reservation categories are guaranteed by the law nº 12.711/2012 together with decrees nº 7.824 e nº 9.034, which were updated by changes in its guidelines in the years 2012, 2016 and 2017. In addition some changes may be the result of the law misinterpretation. These difficulties can result in delays in adhering to legislation, as well as failures in communication between developers and business experts. Therefore, the creation of a language that expresses the distribution rules in a clearer way may assist in the development of processing code with less dependence on technical knowledge in programming, which could contribute to speed up the process of learning and developing new changes on the reservation rules. This work addressed as a methodology, a qualitative research in which 20 users participated, in order to verify the difficulties using the DSL Cotas language, which was developed using the JetBrains Meta Programming System (MPS) tool. In addition, an API was also developed as proof of concept of the language model, with regard to generating valid source code for processing candidates lists. For this purpose, this study takes into account the history of changes in the \gls{IFSC} version control system, in which the main changes in laws versions have been identified so far. As a result of this research, is given a comparative containing the difficulties and concerns about the DSL user experience, obtained between 4 (four) groups of users with distinct academic and experience backgrounds. After this analysis, some improvements in the language were implemented, regarding the users comments and suggestions. Furthermore, with the \gls{API} tests, it was possible to compare the results of the language implementation with the historical processing data present in 16 selective processes. Finally, it is concluded that the DSL Cotas can contribute to the improvement of communication and understanding between the different users knowledges profiles, in addition to proposing a new means of practical-theoretical approach using domain specific languages. The social relevance of this study is emphasized, regarding institutional actions to meet demands that address social inclusion issues as a priority in the students selectives processes.



   \vspace{\onelineskip} 
 
   \noindent 
   \textbf{Key-words}: Domain Specific Language. Meta Programming System. Vacancy Reservation Law 12.711/2012. Candidates Classification Rules. Federal Education Network.
 \end{otherlanguage*}
 \end{resumo}
