
% resumo em português
\begin{resumo}[Resumo] \noindent 
Este trabalho apresenta uma pesquisa que tem como objetivo elaborar uma Domain Specific Language (DSL), que facilite o desenvolvimento de regras de classificação de candidatos conforme o sistema de cotas da rede de ensino pública federal, que é garantida por meio da lei Nº 12.711/2012, em conjunto com os decretos Nº 7.824 e Nº 9.034, os quais passaram por mudanças em suas diretrizes nos anos de 2012, 2016 e 2017. Essas demandas são repassadas pelo \gls{MEC} às instituições, por meio de manuais e instruções normativas, as quais passam por retificações. Adicionalmente a essas mudanças, podem ocorrer situações em que a interpretação de lei é alterada. Essas dificuldades podem resultar em atrasos na aderência à legislação, assim como falhas na comunicação entre desenvolvedores e especialistas de negócio que atuam nos setores responsáveis pelos critérios de aceitação dos sistemas. Portanto, a criação de uma linguagem que expresse regras de distribuição de vagas de maneira mais clara poderá auxiliar na geração de código de classificação com menor dependência de conhecimento técnico em programação, o que pode contribuir para a celeridade dos processos de ingresso em futuras alterações de regras de classificação. Como metodologia propõe-se realizar uma pesquisa de natureza qualitativa, por meio de levantamento histórico de mudanças no sistema de controle de versões do \gls{IFSC} e a análise das alterações realizadas em função de versões de lei até o momento. Como resultado da presente pesquisa será desenvolvida a linguagem proposta na ferramenta Meta Programming System (MPS) da JetBrains. Por fim, a avaliação da pesquisa se dará por meio da verificação de viabilidade sobre o uso da DSL para descrever 3 (três) versões de lei já implementadas no IFSC, assim como a sua utilidade no apontamento de possíveis inconsistências nas regras de classificação antes do início do processo de codificação.
% \noindent %- o resumo deve ter apenas 1 parágrafo e sem recuo de texto na primeira linha, essa tag remove o recuo. Não pode haver quebra de linha.

 \vspace{\onelineskip}
    
 \noindent
 \textbf{Palavras-chaves}: Linguagem de Domínio Específico. Meta Programming System. Lei de Cotas 12.711/2012.
\end{resumo}



% resumo em inglês
\begin{resumo}[Abstract]
 \begin{otherlanguage*}{english}

% \noindent
\lipsum[5]

   \vspace{\onelineskip}
 
   \noindent 
   \textbf{Key-words}: Domain Specific Language. Meta Programming System. Vacancy Reservation Law 12.711/2012.
 \end{otherlanguage*}
\end{resumo}
