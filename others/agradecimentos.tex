% ----------------------------------------------------------
% AGRADECIMENTOS
% ----------------------------------------------------------
\begin{agradecimentos}
Ao Programa de Pós-Graduação em Ciência da Computação do Centro de Informática da Universidade Federal de Pernambuco, em nome do Coordenador Professor \textbf{Ricardo Bastos Cavalcante Prudêncio}.

Ao Orientador Professor \textbf{Leopoldo Motta Teixeira}, pela paciência, por contribuir com os conhecimentos sobre o tema alvo da pesquisa, assim como as suas valiosas contribuições no desenvolvimento do trabalho.

Aos Professores \textbf{Elder de Macedo Rodrigues}, \textbf{Marcio Lopes Cornelio} e \textbf{Vanilson André de Arruda Buregio} pelas contribuições na Banca de Qualificação e de Defesa.

A minha mãe, \textbf{Julia Rosa Severo}, pelo apoio e pela motivação durante o meu processo estudantil e acadêmico.

A minha esposa, \textbf{Ana Paula Boff}, pela compreensão e pelo apoio durante as minhas ausências necessárias para o acompanhamento das aulas e para o desenvolvimento desse estudo.

Aos colegas \textbf{Eliandro Luiz Minski} e \textbf{Samuel Bristot Loli}, pela parceria durante as atividades do curso, assim como pela troca de conhecimentos e experiências de vida que vão contribuir de maneira imensurável para o meu futuro profissional e pessoal.

Ao \textbf{IFSC} pela oportunidade e liberação para as aulas, aos colegas da \textbf{Reitoria}, em especial os amigos da \textbf{DTIC}, pela troca de ideias, dicas, sugestões e pelo apoio nos compromissos e atividades assumidas em função da minha ausência durante as aulas.

A \textbf{todos os participantes} da presente pesquisa, em especial aos colegas do \textbf{Departamento de Ingresso (DEING)} da reitoria e dos campus, que contribuíram imensamente para os levantamentos e testes realizados por esse estudo.

A todos os \textbf{Professores}, \textbf{Técnicos Administrativos} e \textbf{Auxiliares} do \textbf{CIN}, em especial, a servidora \textbf{Joelma Souza de Menezes Franca} que sempre nos atendeu com presteza e prontidão em todos os assuntos e solicitações relativas ao curso.

A \textbf{todos os colegas de aula}, que contribuíram para os trabalhos em grupo, discussões, estudos e risadas durante as atividades realizadas dentro e fora do câmpus. 

Aos membros da \textbf{comunidade MPS Jetbrains}, pelas dúvidas tiradas sobre os recursos do MPS por meio do canal oficial \textbf{slack (\texttt{jetbrains-mps.slack.com})}, que foram de grande auxílio para o desenvolvimento da linguagem DSL Cotas.



\end{agradecimentos}