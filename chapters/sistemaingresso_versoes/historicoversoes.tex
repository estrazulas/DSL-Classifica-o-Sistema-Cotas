\chapter{Sistema de Ingresso e Versionamento}
\label{historicoversoes}

Neste capítulo serão descritas informações sobre as funcionalidades desenvolvidas no sistema de ingresso do \gls{IFSC} com relação aos requisitos e algoritmos do sistema de cotas, para este fim será utilizado o histórico do controle de versão no que concerne ao quantitativo de arquivos, classes, funções/métodos e as diferentes versões desde o surgimento da demanda de cotas na legislação.

\section{Histórico de Projeto}
\label{historicopj}
Criado em meados do ano de 2000 o sistema tem por objetivo disponibilizar vagas de cursos para os discentes do \gls{IFSC}. Este sistema foi desenvolvido internamente na linguagem PHP, para automatizar os processos seletivos, que eram realizados por meio de planilhas e ferramentas não integradas, as quais demandavam ao setor responsável muitas pessoas e muitos procedimentos operacionais repetitivos, gerando falhas no processo por erro humano.

O projeto não utiliza conceitos de orientação a objetos, em sua maioria os arquivos PHP ultrapassam duas mil linhas, sem divisão em camadas \gls{MVC}, com combinações das linguagens Javascript, HTML e PHP no mesmo arquivo. Quando era preciso criar ou adaptar alguma nova funcionalidade, por falta de conhecimento técnico os antigos desenvolvedores (bolsistas) faziam a cópia das funcionalidades para vários locais do sistema, sem pensar em reutilização de código.

Com objetivo de elencar a situação atual do código fonte do sistema, neste trabalho serão apresentados os quantitativos levantados a partir do sistema de controle de versão do \gls{IFSC}. O Quadro \ref{quadro_git_ingresso} apresenta o levantamento geral sobre o total de arquivos, linhas de código, commits e desenvolvedores que já atuaram no projeto. Nas seções seguintes serão contextualizados os dados gerais sobre o algoritmo de classificação, assim como será descrito o levantamento feito nas 3 (três) versões do algoritmo de classificação que foram implementadas até o momento.

\begin{quadro}
\caption{Dados gerais do controle de versão}
\label{quadro_git_ingresso}
\centering
\begin{tabular}{ l l }
   \cline{1-1}\cline{2-2}  
    \multicolumn{1}{|p{5.850cm}|}{\textbf{Total de arquivos}} &
    \multicolumn{1}{p{8.217cm}|}{1.357 arquivos (php, html, css, js )}
  \\ 
   \cline{1-1}\cline{2-2}  
    \multicolumn{1}{|p{5.850cm}|}{\textbf{Total de linhas de código}} &
    \multicolumn{1}{p{8.217cm}|}{36.9414 linhas}
  \\    
   \cline{1-1}\cline{2-2}  
    \multicolumn{1}{|p{5.850cm}|}{\textbf{Total de commits}} &
    \multicolumn{1}{p{8.217cm}|}{731}
  \\    
   \cline{1-1}\cline{2-2}  
    \multicolumn{1}{|p{5.850cm}|}{\textbf{Total de desenvolvedores}} &
    \multicolumn{1}{p{8.217cm}|}{6}
  \\     
   \cline{1-1}\cline{2-2}  
    \multicolumn{1}{|p{5.850cm}|}{\textbf{Período da coleta}} &
    \multicolumn{1}{p{8.217cm}|}{11/02/2015 - 17/04/2019}
  \\       
  \hline

 \end{tabular} 
 \par\medskip\textbf{Fonte:} Sistema de controle de Versão do IFSC \par\medskip
\end{quadro}


\section{Algoritmo de Classificação}
\label{algoritimodeclassificacao}

Neste seção serão detalhados os conceitos e as etapas do processo de classificação de candidatos à vagas do sistema de cotas, assim como o levantamento de cenários exemplifica-tórios sobre distribuição de vagas tendo como base as versões já implementadas no sistema de ingresso.

Para cada processo seletivo há uma lista de cursos, onde o setor que gerencia as vagas define o valor total de vagas iniciais e indica se o processo seletivo vai utilizar a regra de classificação por cotas. Durante as inscrições serão apresentados aos candidatos os campos necessários para indicar que vieram de escola pública e também informar sua renda familiar e se autodeclaram pretos, pardos e indígenas.

Após o término do período de inscrições os candidatos são separados em categorias de concorrência de acordo com o preenchimento realizado no ato da inscrição. A seguir, os candidatos participam do processo seletivo de forma física como provas de vestibular ou processos eletrônicos de classificação, como por exemplo: sorteio, pontuação por preenchimento de formulários sócio-econômicos ou classificação por \gls{ENEM} ou \gls{SISU}. Por fim é gerado um número de classificação que representa a ordem dos candidatos que disputam as vagas disponíveis.

O algoritmo de classificação utiliza como parâmetro de entrada a lista de candidatos, contendo o número de inscrição, a ordem de classificação geral, a data de nascimento (como critério de desempate), a quantidade de vagas total do curso, o percentual de vagas disponível para escola pública e o percentual de proporção do \gls{IBGE}, que varia conforme a Unidade Federada e é fornecido conforme o ultimo censo demográfico, representando o percentual sobre o total de pretos, pardos e indígenas no estado em relação às demais categorias.

Com esses parâmetros de entrada o algoritmo gera um quadro de vagas contendo quantas vagas estão reservadas para às respectivas categorias de cotas, e faz a seleção e aprovação de candidatos de acordo com a sua classificação e critérios de desempate. Por fim, em caso de sobra de vagas por falta de candidatos para uma determinada categoria de cota, o algoritmo utiliza faz nova busca por candidatos de outra categoria, de acordo com a ordem de prioridade estabelecida na portaria Nº 18 de 2012 do \gls{MEC}, a qual define:

\begin{citacao}
Art. 15. No caso de não preenchimento das vagas reservadas aos autodeclarados pretos, pardos
e indígenas e às pessoas com deficiência, aquelas remanescentes serão preenchidas pelos
estudantes que tenham cursado integralmente o ensino fundamental ou médio, conforme o caso,
em escolas públicas, observadas as reservas realizadas em mesmo nível ou no imediatamente
anterior, nos termos do art. 10 desta Portaria. \cite{portarianr9}
\end{citacao}

Ao final do processo de classificação é gerada uma lista de candidatos aprovados, incluindo a classificação geral e a classificação na respectiva categoria de cota, por fim esta lista é enviada ao sistema acadêmico para que os candidatos possam realizar a matrícula e entregar a documentação necessária. 

As alterações nos documentos de lei podem incluir novas categorias, novas formas de distribuição das vagas iniciais, mudanças de percentuais, formas de arredondamento, descrição dos tipos de cotas e mudança na ordem de prioridade em caso de sobra de vagas. Essas situações serão exemplificadas nas seções seguintes por meio dos casos e cenários onde houve maior impacto de refatoração de código do sistema de ingresso para adequação à legislação.


\subsection{Versão 1 - Início do sistema de Cotas, Lei Nº 12.711 de 2012}
\label{versao1}

Inicialmente o sistema de ingresso foi construído sem considerar cotas, ou seja, a classificação era por pontuação ou nota de vestibular, não havia exigência de lei em função de reserva de vagas. As chamadas para matrículas eram realizadas pela ordem geral de classificação e alguns critérios de desempate. Hoje apenas alguns tipos de processos seguem este molde, como por exemplo cursos de curta duração, ou cursos internos que não se aplicam à legislação.

Com o surgimento da lei Nº 12.711/2012 vem também a demanda do governo para que as instituições reservem 50\% das vagas para estudantes do ensino médio exclusivamente em escolas públicas, e dentro destes 50\% deveria ser reservado um percentual para estudantes cuja renda familiar fosse inferior a 1.5 salários mínimos per capita:

\begin{citacao}
Parágrafo único.  No preenchimento das vagas de que trata o caput deste artigo, 50\% (cinquenta por cento) deverão ser reservados aos estudantes oriundos de famílias com renda igual ou inferior a 1,5 salário-mínimo (um salário-mínimo e meio) per capita.

Art. 3o  Em cada instituição federal de ensino superior, as vagas de que trata o art. 1o desta Lei serão preenchidas, por curso e turno, por autodeclarados pretos, pardos e indígenas, em proporção no mínimo igual à de pretos, pardos e indígenas na população da unidade da Federação onde está instalada a instituição, segundo o último censo do Instituto Brasileiro de Geografia e Estatística (IBGE) \cite{leicotas}.
\end{citacao}

Com este objetivo o algoritmo de classificação que era uma consulta SQL foi separado em 3 funções PHP de nome:\textit{calcula\_vagasAcoesAfirmativas}, \textit{aprova\_Candidatos} e \textit{retorna\_OrdemdePreenchimentodeVagasNaoOcupadas}.  A primeira com função de gerar o quadro de vagas, a segunda a seleção dos candidatos para serem aprovados e atualizar a contagem de vagas em cada categoria e a ultima para retornar a ordem de preenchimento em caso de sobra de vagas.

Tendo em vista que há uma divisão categórica das vagas entre os candidatos, foi preciso criar 5 tipos de situações de classificação para cada combinação de cotas possível, essas categorias foram definidas conforme o Quadro \ref{quadro_categoriasv1}.

\begin{table}
\caption{Lista de categorias de cotas da versão 1}
\label{tabela_categoriasv1}
\centering
\begin{tabular}{ l l }
   \cline{1-1}\cline{2-2}  
    \multicolumn{1}{|p{5.850cm}|}{\textbf{CLAG}} &
    \multicolumn{1}{p{8.217cm}|}{Ampla concorrência ou classificação geral 
( todos os candidatos concorrem )}
  \\ 
   \cline{1-1}\cline{2-2}  
    \multicolumn{1}{|p{5.850cm}|}{\textbf{EPRIPPI}} &
    \multicolumn{1}{p{8.217cm}|}{Candidatos que estudaram em Escola Pública com Renda Inferior a 1.5 salários mínimos per capita e autodeclarados Pretos, Pardos ou Indígenas}
  \\    
   \cline{1-1}\cline{2-2}  
    \multicolumn{1}{|p{5.850cm}|}{\textbf{EPRINPPI}} &
    \multicolumn{1}{p{8.217cm}|}{Candidatos que estudaram em Escola Pública com Renda Inferior a 1.5 salários mínimos per capita NÃO são autodeclarados Pretos, Pardos ou Indígenas; }
  \\    
   \cline{1-1}\cline{2-2}  
    \multicolumn{1}{|p{5.850cm}|}{\textbf{EPRSPPI}} &
    \multicolumn{1}{p{8.217cm}|}{Candidatos que estudaram em Escola Pública com Renda Superior a 1.5 salários mínimos per capita e autodeclarados Pretos, Pardos ou Indígenas}
  \\     
   \cline{1-1}\cline{2-2}  
    \multicolumn{1}{|p{5.850cm}|}{\textbf{EPRSNPPI}} &
    \multicolumn{1}{p{8.217cm}|}{Candidatos que estudaram em Escola Pública com Renda Superior a 1.5 salários mínimos per capita NÃO são autodeclarados Pretos, Pardos ou Indígenas}
  \\       
  \hline

 \end{tabular} 
  \par\medskip\textbf{Fonte:} Sistema de controle de versão do IFSC (2015). \par\medskip
\end{table}


\newpage
Dadas as situações de classificação possíveis o primeiro passo do algoritmo é gerar um quadro de vagas para cada tipo de cota, tendo como base o percentual do \gls{IBGE} e o total de vagas para o curso, o trecho de código desenvolvido para chegar a este quadro será apresentado no

\lstinputlisting[language=Octave, 
caption=Octave sample code
]{chapters/trechos_codigo/calculaquadrovagas.m}

\subsection{Versão 2 - Alteração para Lei Nº 13.409 de 2016 }
\label{versao2}

\subsection{Versão 3 - Reimplementação para interpretação do MEC em 2017 }
\label{versao3}

\subsection{Outras customizações realizadas no algoritmo}
\label{outrasVersoes}