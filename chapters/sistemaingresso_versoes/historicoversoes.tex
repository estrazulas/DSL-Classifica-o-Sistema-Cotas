\chapter{Sistema de Ingresso e Versionamento}
\label{chap:historicoversoes}

Neste capítulo são descritas as informações sobre as funcionalidades desenvolvidas no sistema de ingresso do \gls{IFSC} com relação aos requisitos e algoritmos do sistema de cotas. Para esse fim, foi utilizado o histórico do controle de versão no que concerne ao quantitativo de arquivos, classes, funções/métodos e as diferentes versões desde o surgimento da demanda de cotas na legislação.

\section{Histórico de Projeto}
\label{historicopj}
Criado em meados do ano de 2000, o sistema de ingresso do \gls{IFSC}, tem por objetivo disponibilizar vagas de cursos para os discentes do \gls{IFSC}. Esse sistema foi desenvolvido internamente na linguagem PHP, para automatizar os processos seletivos que eram realizados por meio de planilhas e ferramentas não integradas, as quais demandavam ao setor responsável muitas pessoas e muitos procedimentos operacionais repetitivos, gerando falhas no processo por erro humano.

O projeto não utiliza conceitos de orientação a objetos, em sua maioria os arquivos PHP ultrapassam duas mil linhas, sem divisão em camadas \gls{MVC}, com combinações das linguagens Javascript, HTML e PHP no mesmo arquivo. Quando era preciso criar ou adaptar alguma nova funcionalidade por falta de conhecimento técnico, os antigos desenvolvedores (bolsistas) faziam a cópia das funcionalidades para vários locais do sistema, sem pensar em reutilização de código.

Com objetivo de elencar a situação atual do código fonte do sistema, neste trabalho são apresentados os quantitativos levantados a partir do sistema de controle de versão do \gls{IFSC}. A Tabela \ref{quadro_git_ingresso} apresenta o levantamento geral sobre o total de arquivos, linhas de código, commits e desenvolvedores que já atuaram no projeto. Nas Seções seguintes são contextualizados os dados gerais sobre o algoritmo de classificação, assim como é descrito o levantamento feito nas 3 (três) versões do algoritmo de classificação que foram implementadas até o momento.

\begin{quadro}
\caption{Dados gerais do controle de versão}
\label{quadro_git_ingresso}
\centering
\begin{tabular}{ l l }
   \cline{1-1}\cline{2-2}  
    \multicolumn{1}{|p{5.850cm}|}{\textbf{Total de arquivos}} &
    \multicolumn{1}{p{8.217cm}|}{1.357 arquivos (php, html, css, js )}
  \\ 
   \cline{1-1}\cline{2-2}  
    \multicolumn{1}{|p{5.850cm}|}{\textbf{Total de linhas de código}} &
    \multicolumn{1}{p{8.217cm}|}{36.9414 linhas}
  \\    
   \cline{1-1}\cline{2-2}  
    \multicolumn{1}{|p{5.850cm}|}{\textbf{Total de commits}} &
    \multicolumn{1}{p{8.217cm}|}{731}
  \\    
   \cline{1-1}\cline{2-2}  
    \multicolumn{1}{|p{5.850cm}|}{\textbf{Total de desenvolvedores}} &
    \multicolumn{1}{p{8.217cm}|}{6}
  \\     
   \cline{1-1}\cline{2-2}  
    \multicolumn{1}{|p{5.850cm}|}{\textbf{Período da coleta}} &
    \multicolumn{1}{p{8.217cm}|}{11/02/2015 - 17/04/2019}
  \\       
  \hline

 \end{tabular} 
 \par\medskip\textbf{Fonte:} Sistema de controle de Versão do IFSC \par\medskip
\end{quadro}


\section{Algoritmo de Classificação}
\label{algoritimodeclassificacao}

Nesta Seção são detalhados os conceitos e as etapas do processo de classificação de candidatos às vagas do sistema de cotas, assim como o levantamento de cenários ilustrativos sobre a distribuição de vagas tendo como base as versões já implementadas no sistema de ingresso.

Para cada processo seletivo há uma lista de cursos, em que o setor que gerencia as vagas define-as com base no valor total inicial e indica se o processo seletivo vai utilizar a regra de classificação por cotas. Durante as inscrições são apresentados aos candidatos os campos necessários para indicar se vieram de escola pública, bem como informar a sua renda familiar e se autodeclararem pretos, pardos ou indígenas.

Após o término do período de inscrições, os candidatos são separados em categorias de concorrência de acordo com o preenchimento realizado no ato da inscrição. A seguir, os candidatos participam do processo seletivo de forma física como provas de vestibular ou processos eletrônicos de classificação, como por exemplo: sorteio, pontuação por preenchimento de formulários sócio-econômicos ou classificação por \gls{ENEM} ou \gls{SISU}. Por fim é gerado um número de classificação que representa a ordem dos candidatos que disputam as vagas disponíveis.

O algoritmo de classificação utiliza como parâmetro de entrada a lista de candidatos, contendo o número de inscrição, a ordem de classificação geral, a data de nascimento (como critério de desempate), a quantidade de vagas total do curso, o percentual de vagas disponíveis para escola pública e o percentual de proporção do \gls{IBGE}.

Esses parâmetros variam conforme a Unidade Federada e é fornecido pelo último censo demográfico, representando o percentual sobre o total de pretos, pardos e indígenas no estado em relação às demais categorias de cotas para estudantes da escola pública.

Com esses parâmetros de entrada, o algoritmo gera um quadro de vagas contendo quantas vagas estão reservadas para às respectivas categorias de cotas, e faz a seleção e aprovação de candidatos de acordo com a sua classificação e critérios de desempate. Por fim, em caso de sobra de vagas por falta de candidatos para uma determinada categoria de cota, o algoritmo faz nova busca por candidatos de outra categoria, de acordo com a ordem de prioridade estabelecida na portaria Nº 18 de 2012 do \gls{MEC}, a qual define:

\begin{citacao}
Art. 15. No caso de não preenchimento das vagas reservadas aos autodeclarados pretos, pardos
e indígenas e às pessoas com deficiência, aquelas remanescentes serão preenchidas pelos
estudantes que tenham cursado integralmente o ensino fundamental ou médio, conforme o caso,
em escolas públicas, observadas as reservas realizadas em mesmo nível ou no imediatamente
anterior, nos termos do art. 10 desta Portaria. \cite{portarianr9}
\end{citacao}

No fim do processo de classificação é gerada uma lista de candidatos aprovados, incluindo a classificação geral e a classificação na respectiva categoria de cota, por fim esta lista é enviada ao sistema acadêmico para que os candidatos possam realizar a matrícula e entregar a documentação necessária. 

As alterações nos documentos de lei podem incluir novas categorias, novas formas de distribuição das vagas iniciais, mudanças de percentuais, formas de arredondamento, descrição dos tipos de cotas e mudança na ordem de prioridade em caso de sobra de vagas. Essas situações são exemplificadas nas Seções seguintes por meio dos casos e cenários nos quais houve maior impacto de refatoração de código do sistema de ingresso para adequação à legislação.

\subsection{Versão 1 - Início do sistema de Cotas, Lei Nº 12.711 de 2012}
\label{versao1}

Inicialmente, o sistema de ingresso foi construído sem considerar cotas, ou seja, a classificação era por pontuação ou nota de vestibular, não havia exigência de lei em função de reserva de vagas. As chamadas para matrículas eram realizadas pela ordem geral de classificação e alguns critérios de desempate. Hoje apenas alguns tipos de processos seguem este molde, como por exemplo, cursos de curta duração, ou cursos internos que não se aplicam à legislação.

Com o surgimento da lei Nº 12.711 de 2012, vem também a demanda do governo para que as instituições reservem 50\% das vagas para estudantes do ensino médio que estudaram exclusivamente em escolas públicas, e dentro desses 50\% deveria ser reservado um percentual para estudantes cuja renda familiar fosse inferior a 1.5 salários mínimos per capita:

\begin{citacao}
Parágrafo único.  No preenchimento das vagas de que trata o caput deste artigo, 50\% (cinquenta por cento) deverão ser reservados aos estudantes oriundos de famílias com renda igual ou inferior a 1,5 salário-mínimo (um salário-mínimo e meio) per capita.

Art. 3o  Em cada instituição federal de ensino superior, as vagas de que trata o art. 1o desta Lei serão preenchidas, por curso e turno, por autodeclarados pretos, pardos e indígenas, em proporção no mínimo igual à de pretos, pardos e indígenas na população da unidade da Federação onde está instalada a instituição, segundo o último censo do Instituto Brasileiro de Geografia e Estatística (IBGE) \cite{leicotas}.
\end{citacao}

Com esse objetivo, o algoritmo de classificação que era uma consulta SQL foi separado em 3 funções PHP de nome: \textit{calcula\_vagasAcoesAfirmativas}, \textit{aprova\_Candidatos} e \textit{retorna\_OrdemdePreenchimentodeVagasNaoOcupadas}.  A primeira função faz a geração do quadro de vagas, a segunda a seleção dos candidatos para serem aprovados e atualizar a contagem de vagas em cada categoria e a última para retornar a ordem de preenchimento em caso de sobra de vagas.

Tendo em vista que há uma divisão categórica das vagas entre os candidatos, foi preciso criar 5 (cinco) tipos de situações de classificação para cada combinação de cotas possível, essas categorias foram definidas conforme a Tabela \ref{tabela_categoriasv1}.

\begin{table}
\caption{Lista de categorias de cotas da versão 1}
\label{tabela_categoriasv1}
\centering
\begin{tabular}{ l l }
   \cline{1-1}\cline{2-2}  
    \multicolumn{1}{|p{5.850cm}|}{\textbf{CLAG}} &
    \multicolumn{1}{p{8.217cm}|}{Ampla concorrência ou classificação geral 
( todos os candidatos concorrem )}
  \\ 
   \cline{1-1}\cline{2-2}  
    \multicolumn{1}{|p{5.850cm}|}{\textbf{EPRIPPI}} &
    \multicolumn{1}{p{8.217cm}|}{Candidatos que estudaram em Escola Pública com Renda Inferior a 1.5 salários mínimos per capita e autodeclarados Pretos, Pardos ou Indígenas}
  \\    
   \cline{1-1}\cline{2-2}  
    \multicolumn{1}{|p{5.850cm}|}{\textbf{EPRINPPI}} &
    \multicolumn{1}{p{8.217cm}|}{Candidatos que estudaram em Escola Pública com Renda Inferior a 1.5 salários mínimos per capita NÃO são autodeclarados Pretos, Pardos ou Indígenas; }
  \\    
   \cline{1-1}\cline{2-2}  
    \multicolumn{1}{|p{5.850cm}|}{\textbf{EPRSPPI}} &
    \multicolumn{1}{p{8.217cm}|}{Candidatos que estudaram em Escola Pública com Renda Superior a 1.5 salários mínimos per capita e autodeclarados Pretos, Pardos ou Indígenas}
  \\     
   \cline{1-1}\cline{2-2}  
    \multicolumn{1}{|p{5.850cm}|}{\textbf{EPRSNPPI}} &
    \multicolumn{1}{p{8.217cm}|}{Candidatos que estudaram em Escola Pública com Renda Superior a 1.5 salários mínimos per capita NÃO são autodeclarados Pretos, Pardos ou Indígenas}
  \\       
  \hline

 \end{tabular} 
  \par\medskip\textbf{Fonte:} Sistema de controle de versão do IFSC (2015). \par\medskip
\end{table}


\newpage
Dadas as situações de classificação possíveis, o primeiro passo do algoritmo é gerar um quadro de vagas para cada tipo de cota, tendo como base o percentual do \gls{IBGE} e o total de vagas para o curso. O trecho de código desenvolvido será apresentado no Código Fonte \ref{lst:quadrovagas}.


\lstinputlisting[language=PHP, 
caption=Função que calcula o quadro de vagas
,label=lst:quadrovagas]{chapters/trechos_codigo/calculaquadrovagas.m}

Basicamente, o que a função \textit{calcula\_vagasAcoesAfirmativas} faz é separar as vagas conforme os seguintes passos:

\begin{enumerate}
    \item Dado o total de vagas do curso;
    \item Obter o percentual de escola pública;
    \item Obter o percentual de proporção de PPI do IBGE no estado de oferta;
    \item Multiplicar o total de vagas pelo percentual de Escola Pública (50\%) e armazenar o valor em AAEP;
    \item Obter o total reservado para ampla concorrência (CLAG), diminuindo o total de vagas pelo reservado ao sistema de cotas AAEP;
    \item Dividir o total de vagas AAEP em 50\% para candidatos de Renda Inferior a 1.5 (AAEPRI) e 50\% para candidatos de Renda Superior a 1.5 (AAEPRS);
    \item Dentro do total de vagas AAEPRI, deve-se calcular a proporção reservada pelo IBGE para cotistas PPI (Preto, Pardo, Indígenas) que também possuam Renda Inferior a 1.5 salários mínimos e armazenar em AAEPRIPPI;
    \item O total de vagas de estudantes de Renda Inferior que NÃO são autodeclarados PPI (AAEPRINPPI) é obtido após diminuir o total de vagas para renda inferior AAEPRI - o total armazenado em (AAEPRIPPI).
    \item Dentro do total de vagas AAEPRS, deve-se calcular a proporção reservada pelo IBGE para cotistas PPI (Preto, Pardo, Indígenas) que também possuam Renda Superior a 1.5 salários mínimos e armazenar em AAEPRSPPI;
    \item O total de vagas de estudantes de Renda Superior que NÃO são autodeclarados PPI (AAEPRSNPPI) é obtido após diminuir o total de vagas para renda inferior AAEPSI - o total armazenado em (AAEPRSPPI).
\end{enumerate}{}

Um exemplo do resultado da geração do quadro de vagas, para um cenário de curso com 40 vagas,  pode ser visto na Figura \ref{fig:cenario1}.

\begin{figure}[ht!]
\centering

\caption{\textmd{Cenário de distribuição versão 1}}
\label{fig:cenario1}
\fcolorbox{gray}{white}{\includegraphics[width=0.67\textwidth]{chapters/sistemaingresso_versoes/cenarios/cenario1.jpg}}

%\par\medskip\textbf{Fonte:} \cite{bentley} \par\medskip
\end{figure}



\newpage
Tendo com base o quadro de vagas, o algoritmo de classificação de candidatos utiliza a função \textit{aprova\_Candidatos} descrita no Código Fonte \ref{lst:algoritmoaprovacao}, fazendo a seleção dos candidatos para aprovação em cada categoria até atingir o total de vagas disponível. 

\lstinputlisting[language=PHP, 
caption=Função de aprovação de candidatos
,label=lst:algoritmoaprovacao]{chapters/trechos_codigo/aprovacandidatos.m}

Por fim, por motivo de falta de candidatos, o algoritmo utiliza a função \textit{ordemdePreenchimentodeVagasNaoOcupadas} para obter a prioridade por lei quando sobra vaga de uma determinada categoria de cota (Código Fonte \ref{lst:sobravagas}).

\lstinputlisting[language=PHP, 
caption=Função de prioridade em caso de sobra de vagas
,label=lst:sobravagas]{chapters/trechos_codigo/sobravagas.m}

Após análise no controle de versão pode-se identificar que as funções eram utilizadas em diferentes etapas e tipos de processo do sistema. No entanto, a equipe desenvolvedora não teve preocupação na organização do código e fez a cópia em vários pontos do sistema, trazendo ainda mais problemas de manutenção. Os arquivos envolvidos, assim como os dados do controle de versão serão listados nas Tabelas \ref{tabela_arquivosv1_1}, \ref{tabela_arquivosv1_2}, \ref{tabela_arquivosv1_3} e \ref{tabela_arquivosv1_4}.

\newpage
\begin{table}[]
\caption{Versionamento do arquivo corrigir.php}
\label{tabela_arquivosv1_1}
\resizebox{\textwidth}{!}{
\begin{tabular}{@{}|l|l|@{}}
\toprule
\multicolumn{2}{|c|}{ingresso/admin/admin/correcao/corrigir.php}                                        \\ \midrule
\textbf{Total de linhas de código}                               & 1129                                 \\ \midrule
\textbf{Total de funções utilizadas}                             & 267 funções (incluídas e importadas) \\ \midrule
\textbf{Total de funções / algoritmo de cotas}                   & 5 funções                            \\ \midrule
\textbf{Número de linhas envolvidas / algoritmo de cotas}        &                                      \begin{tabular}{@{}|l|l|@{}}
\toprule
\textbf{processar\_Correcao}                             & 96 linhas de código \\ \midrule
\textbf{calcula\_vagasAcoesAfirmativas}                  & 33 linhas de código \\ \midrule
\textbf{aprova\_Candidatos}                              & 73 linhas de código \\ \midrule
\textbf{retorna\_OrdemdePreenchimentodeVagasNaoOcupadas} & 27 linhas de código \\ \midrule
\textbf{alimenta\_Classificacao}                         & 35 linhas de código \\ \bottomrule
\end{tabular}
    \\ \midrule
\textbf{Commits de correções no arquivo}                         & 31 commits desde  11/02/2015         \\ \midrule
\textbf{Número de programadores envolvidos nos commits}          & 2                                    \\ \midrule
\textbf{Total de linhas de código para a implementação de cotas} & 264 linhas                           \\ \bottomrule
\end{tabular}}
\end{table}

\begin{table}[]
\caption{Versionamento do arquivo informar\_classificacao01\_sorteio.php}
\label{tabela_arquivosv1_2}
\resizebox{\textwidth}{!}{
\begin{tabular}{@{}|l|l|@{}}
\toprule
\multicolumn{2}{|c|}{ingresso/admin/admin/semprova/informar\_classificacao01\_sorteio.php}                                        \\ \midrule
\textbf{Total de linhas de código}                               & 1360                                 \\ \midrule
\textbf{Total de funções utilizadas}                             & 236 funções (incluídas e importadas) \\ \midrule
\textbf{Total de funções / algoritmo de cotas}                   & 5 funções                            \\ \midrule
\textbf{Commits de correções no arquivo}                         & 61 commits desde  11/02/2015         \\ \midrule
\textbf{Número de programadores envolvidos nos commits}          & 2                                    \\ \bottomrule
\end{tabular}}
\end{table}

\begin{table}[]
\caption{Versionamento do arquivo informar\_classificacao01.php}
\label{tabela_arquivosv1_3}
\resizebox{\textwidth}{!}{
\begin{tabular}{@{}|l|l|@{}}
\toprule
\multicolumn{2}{|c|}{ingresso/admin/admin/semprova/informar\_classificacao01.php}                                        \\ \midrule
\textbf{Total de linhas de código}                               & 1162                                 \\ \midrule
\textbf{Total de funções utilizadas}                             & 88 funções (incluídas e importadas) \\ \midrule
\textbf{Total de funções / algoritmo de cotas}                   & 6 funções                            \\ \midrule
\textbf{Commits de correções no arquivo}                         & 14 commits desde  11/02/2015         \\ \midrule
\textbf{Número de programadores envolvidos nos commits}          & 2                                    \\ \bottomrule
\end{tabular}}
\end{table}

\begin{table}[h!]
\caption{Versionamento do arquivo  semprova/informar\_classificacao01\_sorteio.php}
\label{tabela_arquivosv1_4}
\resizebox{\textwidth}{!}{
\begin{tabular}{@{}|l|l|@{}}
\toprule
\multicolumn{2}{|c|}{ingresso/admin/admin/semprova/informar\_classificacao01\_sorteio.php}                                        \\ \midrule
\textbf{Total de linhas de código}                               & 1363                                 \\ \midrule
\textbf{Total de funções utilizadas}                             & 119 funções (incluídas e importadas) \\ \midrule
\textbf{Total de funções / algoritmo de cotas}                   & 6 funções                            \\ \midrule
\textbf{Commits de correções no arquivo}                         & 24 commits desde  11/02/2015         \\ \midrule
\textbf{Número de programadores envolvidos nos commits}          & 3                                    \\ \bottomrule
\end{tabular}}
\end{table}

\newpage
Com base nesse levantamento pode-se observar uma quantidade considerável de código fonte envolvido para desenvolvimento de regras para classificação de candidatos. Como tentativa de dar celeridade no processo de desenvolvimento de correções e na criação de futuras versões, os problemas de código duplicado foram reduzidos por meio de refatoração nas versões posteriores do algoritmo de classificação. Na Seção \ref{versao2} são apresentados os dados sobre o controle de versão, pós refatoração realizada em função de novas demandas de alteração da legislação.


\subsection{Versão 2 - Alteração para Lei Nº 13.409 de 2016 }
\label{versao2}


\subsection{Versão 3 - Reimplementação para interpretação do MEC em 2017 }
\label{versao3}


\subsection{Outras customizações realizadas no algoritmo}
\label{outrasVersoes}

