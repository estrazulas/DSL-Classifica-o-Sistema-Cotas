\section{Linguagens Específicas de Domínio}
\label{sec:dsl}

Para \citeonline{kernelif}, a necessidade de abstrações personalizadas sobre um núcleo funcional originam as linguagens de domínio específicos \gls{DSL}. Para \citeonline{spinellis2001notable} \gls{DSL} são, por definição, parte de um sistema maior e geralmente implementado para uso específico de domínio.

\citeonline{mernik2005and} afirmam que as \gls{DSL} fornecem construções sob medida para um domínio específico de aplicativo, fornecendo ganhos substanciais em expressividade e facilidade do uso quando comparado com as \gls{GPL}, o que corresponde em ganhos de produtividade e redução nos custos de manutenção.

No que concerne às questões de comunicação, \citeonline{ghosh2011dsl} descreve que uma \gls{DSL} preenche a lacuna semântica entre usuários de negócio e desenvolvedores, incentivando uma melhor colaboração por meio de vocabulário compartilhado.  

Segundo \citeonline{mernik2005and}, são exemplos de linguagens de domínio específico: HTML, Latex, SQL, Excel, etc. As linguagens possuem domínios de aplicação diferenciados, e a sua aplicação pode reduzir a expertise necessária de programação para seus usuários (Figura \ref{fig:exemplosdsl}).


\begin{figure}[h!]
\centering

\caption{\textmd{Exemplos de DSL amplamente utilizadas}}
\label{fig:exemplosdsl}
\fcolorbox{gray}{white}{\includegraphics[width=0.4\textwidth]{chapters/fundamentacao/imagens/exemplosdsl.jpg}}

\par\medskip\textbf{Fonte:} \citeonline{mernik2005and}. \par\medskip
\end{figure}


As \gls{DSL}s podem ser classificadas de acordo com a sua implementação, se construídas para serem incorporadas em uma \gls{GPL} elas podem ser consideradas como \gls{DSL} internas, enquanto as externas são independentes da infraestrutura de uma linguagem de programação existente \cite{dslengineering}.

Nesse contexto, o presente trabalho objetiva a criação de uma \gls{DSL} externa, com sua própria sintaxe e semântica, de modo que seja possível especificar regras do sistema de cotas da rede de ensino pública federal, de maneira independente à linguagem de programação alvo do sistema. 

\subsection{Diferenças sobre linguagens convencionais}
\label{diferencasdsl}

Segundo \citeonline{dslengineering}, o propósito das \gls{DSL} é de atender a um domínio específico, são construídas para resolver uma classe específica de problemas. De outro lado as \gls{GPL} são voltadas aos desenvolvedores para resolver qualquer tipo de problema computável:

\begin{citacao}
As Linguagens de Programação de Uso Geral (GPLs) são um meio para programadores instruírem computadores.Todas podem ser usadas para implementar qualquer coisa computável com uma máquina de Turing. Isso também significa que qualquer coisa expressável com uma linguagem de programação completa de Turing também pode ser expressa em qualquer outra linguagem de programação completa de Turing. Nesse sentido, todas as linguagens de programação são intercambiáveis. \cite{dslengineering}
\end{citacao}

Para \citeonline{ghosh2011dsl}, projetar uma DSL não é uma tarefa tão assustadora quanto projetar uma linguagem de programação de uso geral. Pois tem foco limitado e é restrita apenas ao domínio que está sendo modelado.

Enquanto \gls{GPL} são flexíveis, as \gls{DSL} sacrificam a flexibilidade em benefício da produtividade e da concisão de programas relevantes em um domínio específico. As \gls{GPL}s são utilizadas em domínios maiores e complexos, do outro lado são trabalhado problemas menores e bem definidos \cite{dslengineering}. Algumas dessas diferenças podem ser observadas na  Figura \ref{fig:gplvsdsl}.

\begin{figure}[h!]
\centering

\caption{\textmd{DSL vs GPL}}
\label{fig:gplvsdsl}
\fcolorbox{gray}{white}{\includegraphics[width=\textwidth]{chapters/fundamentacao/imagens/gplvsdsl.jpg}}

\par\medskip\textbf{Fonte:} \cite{dslengineering} \par\medskip
\end{figure}


Existem situações em que usuários precisam inserir algumas regras  %citar novak



\subsection{Vantagens e desvantagens de uso de DSL}
\label{beneficiosdsl}


\subsection{Ferramentas de construção de DSL}
\label{ferramentasdsl}

