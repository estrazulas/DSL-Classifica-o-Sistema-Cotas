\section{Classificação da pesquisa}
\label{natureza}

A elaboração do presente estudo caracterizou-se como uma pesquisa aplicada quanto à sua natureza. Segundo \citeonline{silva2005pesquisa}, a pesquisa aplicada visa gerar conhecimentos para aplicação prática e voltados à solução de problemas específicos e interesses locais.

Do ponto de vista da abordagem do problema a pesquisa classifica-se como qualitativa. Para \citeonline{silva2005pesquisa}, a pesquisa qualitativa preocupa-se com o processo de investigação, sendo que há uma relação dinâmica entre o objeto a ser estudado e o sujeito, estabelecendo-se um vínculo indispensável entre o mundo objetivo e a subjetividade do sujeito. Somada a essa questão, destaca-se que a interpretação dos fenômenos e a atribuição de significados pelo pesquisador é o instrumento-chave da pesquisa qualitativa.

Em relação aos objetivos, a pesquisa é exploratória, pois buscou:

\begin{citacao}
proporcionar maior familiaridade com o problema, com vistas a torná-lo mais explícito ou a constituir hipóteses[...] seu planejamento é, portanto, bastante flexível, de modo que possibilite a consideração dos mais variados aspectos relativos ao fato estudado \cite[p.41]{gil2002elaborar}.
\end{citacao}

No que concerne aos procedimentos técnicos utilizou-se para o presente estudo: a pesquisa bibliográfica, a documental e o estudo de caso \cite{gil2002elaborar}. Esses procedimentos técnicos referem-se à:

\begin{enumerate}
    \item[a)] Pesquisa bibliográfica: realizada a partir de buscas aos documentos dos principais autores sobre \gls{DSL}: Fowler (2005, 2008) e Voelter (2011, 2013, 2014, 2018). A consulta aos referidos autores possibilitou a identificação de outros materiais sobre a temática, bem como outros pesquisadores utilizados no presente texto. Ademais utilizou-se a legislação sobre as leis de cotas no sistema de ensino público federal, que foram descritas no Capítulo \ref{chap:historicoversoes}. 

    
    \item[b)] Pesquisa documental: utilizada durante a análise do histórico do controle de versão do \gls{IFSC}, mediante a busca de \textit{commits} que contenham a palavra-chave "cotas" e a identificação dos arquivos envolvidos na classificação de candidatos. Nessa análise foram encontrados arquivos contendo as implementações das funcionalidades apresentadas no Capítulo \ref{chap:historicoversoes}, no qual foram detalhados as linhas de código e as funções envolvidas em 3 (três) versões do sistema de ingresso.
    
    \item[c)] Estudo de caso: Para \citeonline{gil2002elaborar}, o estudo de caso tem o propósito de explorar situações da vida real no contexto do objeto estudado, buscando-se analisar e formular hipóteses sobre a sua aplicação. Por esses motivos foi utilizado nessa pesquisa. Para tanto, utilizou-se o estudo de usabilidade da \gls{DSL}, baseado em \citeonline{nielsen2012many}. Para esse autor, para que os testes sejam significativos, pelo menos, 20 usuários precisam participar. Desse modo, foram convidadas 20 pessoas para participar da presente pesquisa respondendo a um exercício proposto para desenvolvimento na \gls{DSL}, bem como um questionário após a realização dos testes. O perfil dos usuários, o detalhamento do exercício e o questionário serão retomados nas próximas seções. 
    
\end{enumerate}


   [TERMINAR] classificados de acordo com as seguintes categorias: 1) não desenvolvedores e especialistas na legislação de cotas; 2) desenvolvedores e não especialistas na legislação de cotas; 3) desenvolvedores e especialistas na legislação de cotas; 4) não desenvolvedores e não especialistas na legislação de cotas. Identificados ao longo do texto respectivamente como: NDEV-ESP; DEV-NESP; DEV-ESP; NDEV-NESP. 