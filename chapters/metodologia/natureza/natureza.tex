\section{Classificação da pesquisa}
\label{natureza}

A elaboração do presente estudo caracterizou-se como uma pesquisa aplicada quanto à sua natureza. Segundo \citeonline{silva2005pesquisa}, a pesquisa aplicada visa gerar conhecimentos para aplicação prática e voltados à solução de problemas específicos e interesses locais.

Do ponto de vista da abordagem do problema a pesquisa classifica-se como qualitativa. Para \citeonline{silva2005pesquisa}, a pesquisa qualitativa preocupa-se com o processo de investigação, sendo que há uma relação dinâmica entre o objeto a ser estudado e o sujeito, estabelecendo-se um vínculo indispensável entre o mundo objetivo e a subjetividade do sujeito. Somada a essa questão, destaca-se que a interpretação dos fenômenos e a atribuição de significados pelo pesquisador é o instrumento-chave da pesquisa qualitativa.

Em relação aos objetivos, a pesquisa é exploratória, pois buscou:

\begin{citacao}
proporcionar maior familiaridade com o problema, com vistas a torná-lo mais explícito ou a constituir hipóteses[...] seu planejamento é, portanto, bastante flexível, de modo que possibilite a consideração dos mais variados aspectos relativos ao fato estudado \cite[p.41]{gil2002elaborar}.
\end{citacao}

No que concerne aos procedimentos técnicos utilizou-se para o presente estudo: a pesquisa bibliográfica, a documental e o estudo de caso \cite{gil2002elaborar}. Esses procedimentos técnicos referem-se à:

\begin{enumerate}
    \item[a)] Pesquisa bibliográfica - sobre o levantamento do estado da arte de conceitos relacionados a modelagem de \gls{DSL}s, presentes em artigos científicos, livros e as principais legislações relacionadas à distribuição de cotas para processos seletivos em instituições públicas da rede de ensino federal;
    
    \item[b)] Pesquisa documental - no que concerne à análise do histórico do controle de versão do \gls{IFSC} relacionando as alterações realizadas em função de documentos de lei ou em função de demandas dos envolvidos no processo de classificação de candidatos, conforme apresentado no Capítulo \ref{chap:historicoversoes};
    
    \item[c)] Estudo de caso - pois tem propósito de explorar a utilização da DSL em contextos reais, buscando-se analisar e formular hipóteses sobre a sua aplicação em diferentes grupos de usuários. Sendo que os participantes do estudo foram classificados em usuários: 1) não desenvolvedores e especialistas na legislação de cotas; 2) desenvolvedores e não especialistas na legislação de cotas; 3) desenvolvedores e especialistas na legislação de cotas; 4) não desenvolvedores e não especialistas na legislação de cotas. Identificados ao longo do texto respectivamente como: NDEV-ESP; DEV-NESP; DEV-ESP; NDEV-NESP.  
    
\end{enumerate}
