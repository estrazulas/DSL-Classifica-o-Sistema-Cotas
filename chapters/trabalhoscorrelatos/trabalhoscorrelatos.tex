\section{Trabalhos Correlatos}
\label{subsection:trabalhoscorrelatos}
 Segundo \citeonline{tahawalid2009}, existem duas maneiras diferentes pelas quais um domínio pode ser definido. A primeira é quando o domínio está bem definido matematicamente, a segunda quando o domínio é definido por atividades puramente humanas. As DSLs de domínio humano, definem uma linguagem especial ou jargões para comunicar ideias relacionadas ao seu domínio.  
 
 Nesse contexto, os trabalhos correlatos encontrados mostraram formas de abstrair conceitos e jargões envolvendo questões legais, contratuais e regras sobre domínios financeiros. Os domínios dessas DSLs foram definidos com objetivo de apoiar na compreensão de atividades humanas e não de tratar de problemas computacionais ou matemáticos.
 
 Para tanto, as DSLs descritas nas próximas Seções foram encontradas por meio da busca pelas palavras-chave (domain specific languages, DSL, law e legal)  nos repositórios, \textit{Researchgate}, \textit{Google Scholar} e \textit{IEEEXplore}.
 
 
\subsection{LegalLanguage}
\label{legallanguage}

A DSL \texttt{LegalLanguage} é resultado da análise de ontologias que tratam de estrutura de documentos legais, sua construção busca estabelecer modelos conceituais com propósito de melhoria de comunicação e aprendizado sobre definições encontradas em textos normativos.

\begin{figure}[ht!]
\centering

\caption{\textmd{Modelo da DSL LegalLanguage}}
\label{fig:legallanguagemodel}
\fcolorbox{gray}{white}{\includegraphics[width=0.90\textwidth]{chapters/trabalhoscorrelatos/imagens/legallanguagemodel.png}}

\par\medskip\textbf{Fonte:} \citeonline{legallanguage}. \par\medskip

\end{figure}



Segundo \citeonline{legallanguage}, o seu modelo é composto de \textit{Laws}, que podem ser classificadas como: Leis internacionais, regulamentos públicos, regulamentos privados, constituições e leis ordinárias. As leis são compostas por uma série artigos ordenados sequencialmente, os quais são organizados em divisões. É possível criar relacionamentos hierárquicos entre artigos de diferentes elementos de lei, bem como indicar revogações quando novas leis surgem (Figura \ref{fig:legallanguagemodel}). 

Segundo \citeonline[p.45, tradução nossa]{legallanguage}: "O desenvolvimento dessa linguagem tem como objetivo apoiar parlamentares que executam atividades inseridas em processos legislativos de redação de leis". Portanto, os autores à criaram utilizando a ferramenta \texttt{Xtext}, um exemplo de definição e uso do elemento \texttt{Law} pode ser observado nas Figuras \ref{fig:xtextlegal} e \ref{fig:legallanguageexample}.

\begin{figure}[ht!]
\centering

\caption{\textmd{Definição do elemento Law na LegalLanguage}}
\label{fig:xtextlegal}
\fcolorbox{gray}{white}{\includegraphics[width=0.85\textwidth]{chapters/trabalhoscorrelatos/imagens/xtextlegal.png}}

\par\medskip\textbf{Fonte:} \cite{legallanguage}. \par\medskip

\end{figure}




\begin{figure}[ht!]
\centering

\caption{\textmd{Exemplo ilustrativo da LegalLanguage}}
\label{fig:legallanguageexample}
\fcolorbox{gray}{white}{\includegraphics[width=0.90\textwidth]{chapters/trabalhoscorrelatos/imagens/legallanguageexample.png}}

\par\medskip\textbf{Fonte:} \cite{legallanguage}. \par\medskip

\end{figure}





\newpage
\subsection{Ergo uma \textit{DSL} para acordos Legais}
\label{ergo}


\subsection{DSLFIN uma \textit{DSL} para sistemas financeiros}
\label{dslfin}