\section{Trabalhos Relacionados}
\label{subsection:trabalhoscorrelatos}
 Segundo \citeonline{tahawalid2009}, existem duas maneiras diferentes pelas quais um domínio pode ser definido. A primeira é quando o domínio está bem definido matematicamente, a segunda quando o domínio é definido por atividades puramente humanas. As DSLs de domínio humano, definem uma linguagem especial ou jargões para comunicar ideias relacionadas ao seu domínio.  
 
 Assim como a DSL Cotas, os trabalhos relacionados encontrados mostraram formas de modelar e abstrair conceitos e jargões envolvendo questões legais, contratuais e regras sobre domínios financeiros. Os domínios dessas DSLs foram definidos com objetivo de apoiar na compreensão de atividades humanas e não de tratar de problemas computacionais ou matemáticos.
 
 Para tanto, as DSLs descritas nas próximas Seções foram encontradas por meio da busca pelas palavras-chave (domain specific languages, DSL, law e legal)  nos repositórios, \textit{Researchgate}, \textit{Google Scholar} e \textit{IEEEXplore}.
 
 
\subsection{LegalLanguage}
\label{legallanguage}

A DSL \texttt{LegalLanguage} é resultado da análise de ontologias que tratam de estrutura de documentos legais, sua construção busca estabelecer modelos conceituais com propósito de melhoria de comunicação e aprendizado sobre definições encontradas em textos normativos.

Apesar de tratar de domínio de questões legais, sua modelagem difere do presente trabalho pois sua construção foi baseada em ontologias já existentes, com propósito de auxiliar em definições de documentos legais de maneira geral, enquanto a DSL Cotas foi modelada a partir do histórico de controle de versão entre as diferentes versões da legislação e trata especificamente da lei de cotas. 

\begin{figure}[ht!]
\centering

\caption{\textmd{Modelo da DSL LegalLanguage}}
\label{fig:legallanguagemodel}
\fcolorbox{gray}{white}{\includegraphics[width=0.90\textwidth]{chapters/trabalhoscorrelatos/imagens/legallanguagemodel.png}}

\par\medskip\textbf{Fonte:} \citeonline{legallanguage}. \par\medskip

\end{figure}



Segundo \citeonline{legallanguage}, o seu modelo é composto de \textit{Laws}, que podem ser classificadas como: Leis internacionais, regulamentos públicos, regulamentos privados, constituições e leis ordinárias. As leis são compostas por uma série de artigos ordenados sequencialmente, os quais são organizados em divisões. É possível criar relacionamentos hierárquicos entre artigos de diferentes elementos de lei, bem como indicar revogações quando novas leis surgem (Figura \ref{fig:legallanguagemodel}). 

Segundo \citeonline[p.45, tradução nossa]{legallanguage}: "O desenvolvimento dessa linguagem tem como objetivo apoiar parlamentares que executam atividades inseridas em processos legislativos de redação de leis". Portanto, os autores a criaram utilizando a ferramenta \texttt{Xtext}, um exemplo de definição e uso do elemento \texttt{Law} pode ser observado nas Figuras \ref{fig:xtextlegal} e \ref{fig:legallanguageexample}.

\begin{figure}[ht!]
\centering

\caption{\textmd{Definição do elemento Law na LegalLanguage}}
\label{fig:xtextlegal}
\fcolorbox{gray}{white}{\includegraphics[width=0.85\textwidth]{chapters/trabalhoscorrelatos/imagens/xtextlegal.png}}

\par\medskip\textbf{Fonte:} \cite{legallanguage}. \par\medskip

\end{figure}




\begin{figure}[ht!]
\centering

\caption{\textmd{Exemplo ilustrativo da LegalLanguage}}
\label{fig:legallanguageexample}
\fcolorbox{gray}{white}{\includegraphics[width=0.90\textwidth]{chapters/trabalhoscorrelatos/imagens/legallanguageexample.png}}

\par\medskip\textbf{Fonte:} \cite{legallanguage}. \par\medskip

\end{figure}





\newpage
\subsection{Ergo uma DSL para Contratos Legais Inteligentes}
\label{ergo}

Ergo é uma linguagem específica de domínio que faz parte do projeto Accord, o qual define a base jurídica e técnica para contratos legais inteligentes, com objetivo de atender aos problemas relacionados à falta de padronização em contratos legais. 

Segundo \citeonline{accordproject}, a linguagem tem o foco de ajudar desenvolvedores da área de tecnologia jurídica na escrita de contratos legais que possam ser computados. Um contrato legal inteligente é um contrato legível por seres humanos e por máquinas, por exemplo, uma cláusula de cobrança de pagamento pode estar presente em um contrato, de modo que se possa utilizar o texto descrito em linguagem natural, para extração dos dados de cobrança e posterior aplicação de cálculos de multas e geração de eventos de notificações. 

De forma similar a DSL Cotas, sua modelagem garante que as definições dos usuários sejam escritas com expressividade limitada, por outro lado, enquanto a DSL Cotas utiliza a definição das regras de distribuição de cotas em formato de tabelas, a Ergo DSL utiliza um formato textual controlado, similar ao de estrutura de classes para contratos e métodos para suas cláusulas.

É uma linguagem fortemente tipada e independente de plataforma, seu código pode ser compilado nas plataformas \texttt{JavaScript} e \texttt{Java}. Ela provê uma linguagem de expressões, pela qual é possível descrever funções e cláusulas de modo que possa ser definida a lógica de contratos inteligentes. 

Um exemplo de linguagem natural pode ser observado na Figura \ref{fig:ergotexto}, no qual os dados sobre a cláusula de cobrança são mapeados para posteriormente serem modelados e terem a lógica de cobrança seja definida (Figuras \ref{fig:ergomodelo} e \ref{fig:ergologica}). 

\begin{figure}[ht!]
\centering

\caption{\textmd{Exemplo ilustrativo da Ergo DSL}}
\label{fig:ergotexto}
\fcolorbox{gray}{white}{\includegraphics[width=\textwidth]{chapters/trabalhoscorrelatos/imagens/ergotexto.PNG}}

\par\medskip\textbf{Fonte:} \citeonline{accordproject}. \par\medskip

\end{figure}



\begin{figure}[ht!]
\centering

\caption{\textmd{Modelagem de clausula na ERGO DSL}}
\label{fig:ergomodelo}
\fcolorbox{gray}{white}{\includegraphics[width=0.90\textwidth]{chapters/trabalhoscorrelatos/imagens/ergomodelo.PNG}}

\par\medskip\textbf{Fonte:} \citeonline{accordproject}. \par\medskip

\end{figure}



\begin{figure}[ht!]
\centering

\caption{\textmd{Definição da lógica de contratos}}
\label{fig:ergologica}
\fcolorbox{gray}{white}{\includegraphics[width=0.90\textwidth]{chapters/trabalhoscorrelatos/imagens/ergologica.PNG}}

\par\medskip\textbf{Fonte:} \citeonline{accordproject}. \par\medskip

\end{figure}



\newpage

\subsection{DSL Capgemini Pension}
\label{capgeminipension}

Segundo \citeonline{gregfuller2013}, foi criada com o objetivo de atender ao domínio de planos de previdência e fornecimento de planos de seguros de pensão, no qual cada plano pode conter centenas de regras específicas, tratadas conforme o histórico de até 40 anos de dados funcionais para milhares de segurados. 

 Para \citeonline{kolkhenk2008}, essa linguagem advém da necessidade de inovação em produtos da área e é baseada em interesses governamentais, no sentido de que há a necessidade de adequação às novas leis de pensão, dar transparência sobre as regras e fornecer garantia de qualidade. 

Por tanto, essa DSL se relaciona ao presente trabalho no que diz respeito à necessidade de adequação, inovação e evolução para questões de interesses governamentais e para atendimento de novas legislações.

Criada com a ferramenta \texttt{Intentional Software} para construção de DSLs projecionais, ela permite que as definições das regras dos planos sejam realizadas em formato gráfico no estilo de tabelas de Excel (Figura \ref{fig:dslcapgeminitables}) e permite fazer testes unitários para validar as regras em tempo real (Figura \ref{fig:dslcapgeminiunittests}). 

\begin{figure}[ht!]
\centering

\caption{\textmd{Editor projecional das regras de pensão}}
\label{fig:dslcapgeminitables}
\fcolorbox{gray}{white}{\includegraphics[width=0.95\textwidth]{chapters/trabalhoscorrelatos/imagens/dslcapgeminitables.PNG}}

\par\medskip\textbf{Fonte:} \citeonline{kolkhenk2008}. \par\medskip

\end{figure}



\begin{figure}[ht!]
\centering

\caption{\textmd{Testes unitários das regras}}
\label{fig:dslcapgeminiunittests}
\fcolorbox{gray}{white}{\includegraphics[width=\textwidth]{chapters/trabalhoscorrelatos/imagens/dslcapgeminiunittests.png}}

\par\medskip\textbf{Fonte:} \cite{kolkhenk2008}. \par\medskip

\end{figure}



O ambiente da linguagem substitui uma série de planilhas, documentos de texto e outras fontes de controle utilizadas por múltiplos especialistas de domínio no contexto de gerenciamento dos planos de pensão. Dessa forma, a \textit{Pension Workbench} traz um ambiente colaborativo controlado, com sistema integrado de controle de versão, sendo passível criar testes unitários e gerar a \textit{engine} responsável pelos cálculos para a administração dos planos e fundos de pensão \cite{gregfuller2013}.

Portanto, esses trabalhos relacionados assemelham-se à presente pesquisa no sentido de abordarem DSLs como meio de abstração de conhecimento de negócio. Eles visam a melhoria de processos que necessitem de maior clareza por parte dos envolvidos, criando um ambiente controlado que possa favorecer ao desenvolvimento do produto desejado pelos usuários especialistas. 

Considerando a DSL desenvolvida por esse estudo, a seguir serão apresentados: o escopo negativo da pesquisa, as suas principais contribuições e os trabalhos futuros sugeridos.
