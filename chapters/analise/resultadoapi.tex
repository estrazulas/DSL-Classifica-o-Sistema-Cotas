\section{Resultados obtidos com a API DSL Cotas}
\label{sec:avaliacaoapi}

Nessa Seção são descritos os resultados provenientes dos testes da API DSL Cotas, nos quais tiveram como objetivo principal a comparação do histórico de processos seletivos do \gls{IFSC} com o resultado gerado pela API, de modo a verificar a conformidade de seus resultados em relação às legislações vigentes e/ou mais antigas.

Para a realização dos testes da API, o autor da presente pesquisa replicou as regras e as siglas equivalentes da primeira e da versão mais recente da legislação na DSL Cotas. Foi necessário inicializar a lista de candidatos na base de dados com os parâmetros iniciais antes da classificação, de modo a simular o processo de primeira chamada ocorrido na época.

Essa análise contou com a seleção aleatória de 16 processos seletivos, incluindo 403 cursos do tipo: técnico integrado, concomitante, subsequente, proeja e graduação. Nesse contexto, foram considerados os resultados de 13494 candidatos no período de 2013 até 2020, nos quais foi possível testar 2 (duas) das versões já processadas pelo sistema de ingresso, além das demais alterações de código ocorridas no período. 

Não foi possível fazer o comparativo com a versão intermediária descrita na Subseção \ref{versao2} do Capítulo \ref{chap:historicoversoes}, uma vez que não houve processamento de resultados na época, pois o código precisou ser atualizado com a versão mais recente disponibilizada pelo MEC, antes da sua utilização nos processos seletivos.

Como resultado dos testes foi gerado um relatório contendo as colunas, "Versão de lei utilizada", "Edital", "Identificador do Curso", "Número de vagas", "Código de Inscrição", "Classificação", "Categoria esperada" e "Resultado da conferência". Para cada candidato foi acionado o \textit{endpoint} \texttt{aprovaCandidatos} em classe de testes do \texttt{JUnit}, na qual foram realizadas requisições \texttt{HTTP} passando a lista de inicial de candidatos inscritos, sem a situação de classificação.

A listagem resultante da API passou por comparação de todas as siglas de situação de classificação conforme o sistema de cotas, sendo possível retornar e comparar a sigla de situação original presente na base do sistema de ingresso. Desse modo, foi verificado que 81 dos 403 cursos apresentaram divergências nas listas de classificação de candidatos.

Essas divergências foram marcadas em 297 candidatos dos 13494 registros utilizados. Para cada curso com divergência foi realizada uma análise manual que teve como objetivo identificar o problema, o motivo e uma possível solução. Essas divergências foram agrupadas nas seguintes \textit{issues} cadastradas no repositório \texttt{https://github.com/estrazulas/dsl-cotas-gen/issues}:

\begin{enumerate}
    \item[a)] \textbf{Candidato sem situação de classificação}: Situação encontrada em 2 (dois) cursos, o problema apresenta alguns candidatos sem sigla de categoria na lista resultante com o uso da API. Nos 2 (dois) casos, o motivo da divergência aponta para o fato de o teste ter sido realizado apenas em candidatos de primeira chamada, sendo que os candidatos com problema foram convocados posteriormente pelo sistema de ingresso. Não sendo um problema a ser solucionado pela API, uma vez que em chamadas posteriores as situações seriam recalculadas;
    
    \item[b)] \textbf{Cursos com candidatos em rechamada}: Em 31 dos cursos testados foram encontrados candidatos em situação de inscrição "REC" ou re-chamados, no entanto, a seleção de candidatos para aplicação do testes apenas considerou a busca por candidatos de primeira chamada e em situação "CLA", sigla utilizada antes da etapa de aprovação e atribuição de categorias. Os candidatos re-chamados faziam parte de uma regra de negócio antiga do sistema, que dava a oportunidade de serem reconvocados para matrícula em chamadas posteriores, e por esse motivo não entraram no filtro de comparação de candidatos em primeira chamada;
    
    \item[c)] \textbf{Candidato cotista convocado como ampla concorrência}: Em apenas um curso foi encontrado um candidato de inscrição por cotas que na API foi marcado para a categoria correta, no entanto, no sistema de ingresso consta com a categoria de ampla concorrência. Após análise manual identificou-se que seguindo as regras de processamento, o candidato da posição 18 deveria ter sido convocado como cotista. No entanto, a causa dessa divergência é desconhecida, sendo possível que na época esse candidato tenha sido alterado manualmente em base de dados;
    
      
    \item[d)] \textbf{Candidato da ampla concorrência classificado como cotista}: A divergência com maior número de casos (42 cursos do \gls{SISU}). Após análise no processamento foi identificado que os candidatos inscritos como cotistas com pontuação superior aos de candidatos da ampla concorrência (CLAG) foram aprovados pelo \gls{SISU} como cotistas, quando em todos os demais processos seletivos do sistema o funcionamento sugere que os primeiros colocados sejam selecionados e classificados pela categoria CLAG. Em síntese, o processamento foi diferente ao adotado pelo SISU, o que não pode ser resolvido no contexto da presente pesquisa.
    
\end{enumerate}

Destaca-se que as divergências encontradas foram provenientes de situações ocorridas durante o andamento dos processos de inscrições de candidatos, não sendo possíveis de serem tratadas pela DSL Cotas, uma vez que o próprio sistema de ingresso modificou as situações de classificações na medida em que outras chamadas foram sendo realizadas.

Nos demais casos de divergência foram identificados candidatos com problemas relacionados a operações na base de dados em função de abertura de chamados do setor demandante, ou situações que puderam ser identificadas e resolvidas por meio de correção do código da API. O detalhamento completo das divergências, assim como o relatório utilizado para a análise estão disponíveis no repositório \textit{github} citado anteriormente.

Por fim, essa análise sugere que na maioria dos casos de cursos testados foi possível combinar o formalismo definido pela DSL Cotas em conjunto com a API, de modo a validar os diferentes tipos de cursos e versões da legislação já utilizadas pelo \gls{IFSC}. A sua utilização pode favorecer a produtividade e a evolução de novas alterações em lei, de maneira agnóstica sem estar atrelada a uma \gls{GPL} específica, uma vez que sua construção foi concebida a nível de serviços \textit{web}.