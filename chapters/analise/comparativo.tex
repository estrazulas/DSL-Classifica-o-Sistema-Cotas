\subsection{Comparativo entre os grupos de análise}
\label{sec:comparativogrupos}

Em relação aos critérios estabelecidos para a análise, considerando o levantamento apresentado pelas Figuras \ref{fig:quadro:grupodevesp}, \ref{fig:quadro:grupodevnesp}, \ref{fig:quadro:grupondevesp} e \ref{fig:quadro:grupondevnesp}, observou-se que o grupo DEV-ESP teve mais facilidade de entendimento e utilização dos recursos da DSL, uma vez que todos os usuários conseguiram definir as regras propostas pelo exercício, além das soluções apresentarem poucos \textit{errors} e \textit{warnings}. Esse fato reforça que: "o uso de DSLs com entendimento do domínio, deixa o pensamento ser expresso de maneira mais clara quando o código escrito não está repleto de detalhes de implementação" \cite[p.41, tradução nossa]{dslengineering}.

Em segundo lugar, considerando a correta implementação das regras propostas, o grupo NDEV-ESP apenas 2 (dois) dos 8 (oito) usuários não conseguiram configurar a distribuição completa, no entanto, 6 (seis) usuários fizeram as definições conforme o exercício, sendo que 2 (dois) deles avançaram no desenvolvimento da versão mais recente e complexa da legislação, contemplando candidatos inscritos na categoria PCD. 

O fato dos usuários NDEV-ESP terem conseguido atender uma legislação diferente da proposta, pode se relacionar com um dos benefícios de DSLs: 

\begin{citacao}
O uso de DSLs específicas de domínio, podem parecer, a primeiro momento, difícil de se justificar, porém essas DSLs são normalmente atreladas ao \textit{know-how} do negócio, provendo um meio de descrever conhecimento de maneira formal, organizada e sustentável. \cite[p.43, tradução nossa]{dslengineering}.
\end{citacao}

Em continuidade a esse comparativo, o grupo DEV-NESP ficou em terceiro lugar no que diz respeito ao entendimento sobre os recursos utilizados, esses usuários apontaram no questionário um grau de dificuldade elevado, o que foi agravado pelo fato de desconhecerem a área de domínio. Contudo, apesar das dificuldades, na maioria dos casos, todos os recursos foram utilizados sem que fossem gerados muitos erros ou avisos. 

Por fim, a análise do grupo NDEV-NESP mostra que há potencial de uso da DSL, no entanto, destaca-se que há necessidade de mais explicações sobre as regras de domínio, assim como de capacitação para uso na linguagem. Isso se justifica, pelo fato de que esse grupo apresentou maior quantidade de erros além de alguns exercícios terem sido entregues de forma incompleta.

