\begin{table}[ht]
\caption{\textit{Quick fixes da DSL Cotas}}
\label{tblquickfixes}
\centering

\begin{tabular}{|p{6cm}|p{9cm}|}
\hline
\texttt{preenche\_totalvagas\_name} & \textit{Quick fix} associado ao erro gerado pela \textit{checking rule} \texttt{distribuicao\_nome}, faz o preenchimento automático do nome padrão no ramo raiz de distribuição de vagas.                                                                                      \\ \hline
\texttt{remover\_categoria\_duplicada} & \textit{Quick fix} associado ao erro gerado pela \textit{checking rule} \texttt{categoria\_unica}, sugere a remoção automática do ramo de distribuição com a duplicidade.

\\ \hline
\texttt{reserva\_vagas\_ultima\_da\_lista} & \textit{Quick fix} associado ao erro gerado pela \textit{checking rule} \texttt{categoria\_resestante\_vagas}, sugere a correção automática da distribuição de vagas para preenchimento adequado da constante \texttt{RESTANTE\_VAGAS}.

\\ \hline
\texttt{sugere\_sigla\_nome} & \textit{Quick fix} associado ao \textit{warning} gerado pela \textit{checking rule} \texttt{warning\_nm\_inic\_cota}, sugerindo a correção automática para padronização do nome das siglas de distribuição.
                  \\ \hline      

\end{tabular}

  \par\medskip\textbf{Fonte:} Elaborada pelo autor (2020). \par\medskip
\end{table}

