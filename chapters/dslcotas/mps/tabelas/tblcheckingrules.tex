\begin{table}[ht]
\caption{\textit{Checking rules} da DSL Cotas}
\label{tblcheckingrules}
\centering

\begin{tabular}{|p{6cm}|p{9cm}|}
\hline
\texttt{categoria\_resestante\_vagas} & Busca identificar se o último ramo de uma divisão de vagas é preenchida pelo usuário com a constante \texttt{RESTANTE\_VAGAS}.                                                                                      \\ \hline
\texttt{categoria\_unica} & Impede siglas de categorias de cotas duplicadas.                         \\ \hline
\texttt{codigo\_versao\_lei}          & Valida padrões de preenchimento dos dados de identificação da lei de cotas.                                       \\ \hline
\texttt{configuracao\_simples}          & Impede que o usuário preencha expressões complexas em uma definição do conceito \texttt{Configuracao}.
                        \\ \hline
\texttt{divisao\_pelo\_menos\_dois\_ramos}          & Garante que uma divisão de categoria possua pelo menos 2 (duas) categorias, exceto o ramo raiz.
                        \\ \hline
\texttt{filha\_sem\_reserva}          & Verifica se todas as categorias possuem o campo de percentual de reserva preenchido.

\\ \hline
               
\texttt{formato\_sigla\_cota}          & Faz o \textit{matching} da \texttt{String} de siglas de cota, permitindo apenas caracteres alfanuméricos maiúsculos.
\\ \hline

\texttt{distribuicao\_nome}          & Garante que o ramo raiz \texttt{TOTAL\_VAGAS} não tenha o nome alterado pelo usuário da DSL.
\\ \hline

\texttt{ordem\_prioridade\_duplicada}          & Valida se o usuário informou uma categoria de cota duplicada no conceito \texttt{OrdemPrioridadeCotas}.
\\ \hline

\texttt{ordem\_prioridade\_catfilhas}          & Gera um \textit{warning} para o usuário quando uma categoria não foi informada na seção \texttt{OrdemPrioridadeCotas}.
\\ \hline

\texttt{warning\_nm\_inic\_cota}          & Sugere um nome de preenchimento para a sigla de cota que inicie com o prefixo da cota anterior, por exemplo: a cota de estudantes de renda inferior (sigla RI), que fica dentro da categoria de escola pública (sigla EP), tem como sugestão EP\_RI, para facilitar a identificação.
\\ \hline

\end{tabular}

  \par\medskip\textbf{Fonte:} Elaborada pelo autor (2020). \par\medskip
\end{table}
