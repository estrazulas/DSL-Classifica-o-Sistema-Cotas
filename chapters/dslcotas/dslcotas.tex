\chapter{A DSL DE COTAS}
\label{chap:dslcotas}

   Neste Capítulo é descrita a linguagem desenvolvida na ferramenta \gls{MPS}, que possui como principal objetivo facilitar a definição das regras de distribuição de cotas e, dessa forma, permitir que os usuários do sistema de ingresso do \gls{IFSC} possam alterar essas definições de forma independente à implementação do respectivo algoritmo de classificação.
   
   Como foi descrito no Capítulo \ref{chap:historicoversoes}, desde a promulgação da lei de cotas em 2012, o sistema de ingresso passou por, pelo menos, 3 (três) versões diferentes de distribuição de vagas, ocasionando em várias refatorações de código desde então. 
   
   Desse modo, a presente pesquisa tem como foco: a definição de uma linguagem para descrever todos os 3 (três) cenários passados, com apoio de recursos de linguagens existentes no \gls{MPS} tais como: validação de estrutura e de tipos de dados aceitos, preenchimento automático de código de acordo com o contexto, validação de inconsistências, recursos de dicas ao usuário da linguagem e geração de código fonte responsável pela classificação e aprovação de candidatos. 
   
   Com apoio desses recursos, o usuário especialista no sistema de cotas, poderá fazer o uso controlado da especificação de regras, assim como, no caso de surgimento de um novo cenário de distribuição de vagas. Esse conjunto de regras alteradas podem ser transformadas em código fonte, possibilitando a  sua importação no sistema de ingresso do \gls{IFSC}, diminuindo a necessidade de refatoração em seu código.
   
   Nesse sentido, a Seção \ref{sec:dslproposta:usuario} aborda as características de implementação da \gls{DSL} a ser utilizada por especialistas de negócio, para definição da árvore de distribuição de cotas, tendo como base as especificações de lei. Enquanto na Seção \ref{apicotas}, apresenta-se o detalhamento do desenvolvimento da \gls{API} responsável por classificar e aprovar candidatos tendo como base a DSL Cotas.
   
  
   \section{A Modelagem da DSL Cotas}
\label{sec:dslproposta:usuario}

   
 Tendo como base a análise realizada no Capítulo \ref{chap:historicoversoes}, sobre as versões do sistema de cotas detalhadas nas Subseções \ref{versao1}, \ref{versao2} e \ref{versao3}, podem ser  pontuadas algumas características em comum identificadas entre cada uma dessas versões:
   
   \begin{enumerate}
    \item[a)] A divisão das vagas entre as diferentes categorias de cotas se deu em formato hierárquico, iniciando no total de vagas o qual foi sendo subdividido em percentuais reservados para suas subcategorias (ex: Estudantes da escola pública, Estudantes com renda baixa, Estudantes PCD, etc); 
   
   \item[b)] Esses percentuais são de conhecimento dos usuários especialistas no sistema de cotas, e podem variar de acordo com a documentação da legislação vigente, ou outras definições que variam conforme a Unidade Federada que oferta o curso; 
   
   \item[c)] Em muitos casos, não é definido um valor de percentual fixo, de modo que uma categoria recebe o valor calculado restante de vagas da categoria pai. Por exemplo, o percentual de cotistas \gls{PCD} é aplicado, e o que resta das vagas vai para candidatos da mesma categoria que não são \gls{PCD};

   \item[d)] Todas as versões consideravam a ordem de prioridade entre as diferentes categorias de cotas passíveis de inscrição;
   
   \item[e)] Questões de arredondamento das vagas (para cima ou para baixo), devem ser consideradas e podem variar de acordo com a versão de lei implementada.

   \end{enumerate}
   
   Segundo \citeonline{dslengineering}, a modelagem de DSLs costuma ser realizada com abordagens \textit{bottom-up} ou \textit{top-down}, a primeira abordagem aplica-se para domínios em termos de software, enquanto na segunda, o modelo do domínio é aplicado ao conhecimento sobre o mundo real, fora do domínio de software. Nesse contexto, a modelagem da DSL Cotas utilizou a abordagem \texttt{top-down}, uma vez que seu design procura fornecer suporte para os usuários especialistas definirem os parâmetros da legislação de forma controlada, possibilitando a geração do código de processamento de classificação de candidatos sem que o usuário precise utilizar comandos complexos para a implementação dos respectivos algoritmos.
   
   A \gls{DSL} desenvolvida consistiu em criar um modelo que permita a configuração de todos os itens listados, tais como: identificador da versão de lei, lista de variáveis para configuração de percentuais, formas de aplicação de arredondamento, uma macro para aplicar a função de resto de vagas de uma categoria, campo descritivo para os diferentes tipos de categorias, estrutura para divisão em subcategorias e lista de ordem de prioridade para a sobra de vagas.
   
   
   
    Na Seção \ref{sec:mps}, são apresentados os componentes desenvolvidos no MPS que são responsáveis pela construção da DSL Cotas.
    
   \section{Componentes da DSL Cotas no \gls{MPS}}
\label{sec:mps}

Nessa Seção são descritos os elementos de modelagem criados na ferramenta \gls{MPS} tais como: os componentes de estrutura (\texttt{structure concepts}), a sintaxe dos editores projecionais (\texttt{editors}), restrições de escopo (\texttt{constraints}), comportamentos de conceitos (\texttt{behaviors}), sistema de tipos (\texttt{typesystem}) e, por fim, os elementos de geração de texto (\texttt{textGen}).

\subsection{\textit{Componentes de estrutura}}
\label{sub:sec:estrutura}

\textit{Concepts} ou conceitos no \gls{MPS} servem para definir a estrutura base da linguagem, cada conceito pode conter propriedades, outros conceitos filhos \texttt{childrens} e referências para outros conceitos. Eles podem herdar ou implementar características de outros conceitos. A Figura \ref{fig:structure} apresenta a lista de conceitos criados para a DSL Cotas.

\begin{figure}[ht!]
\centering

\caption{\textmd{Lista de Conceitos de Estrutura \gls{MPS}}}
\label{fig:structure}
\fcolorbox{gray}{white}{\includegraphics[width=0.75\textwidth]{chapters/dslcotas/mps/imagens/structure.png}}

\par\medskip\textbf{Fonte:} Elaboração do autor (2020) \par\medskip

\end{figure}





Esses conceitos definem a estrutura hierárquica da \gls{AST} de modo análogo ao modelo orientado a objetos, portanto, a Figura \ref{fig:classesmps} mostra a representação da modelagem dos conceitos em formato de diagrama de classes da \gls{UML}.

\begin{figure}[ht!]
\centering

\caption{\textmd{Modelo de Conceitos no \gls{MPS}}}
\label{fig:classesmps}
\fcolorbox{gray}{white}{\includegraphics[width=\textwidth]{chapters/dslcotas/mps/imagens/classesmps.png}}

\par\medskip\textbf{Fonte:} Elaborada pelo autor (2020). \par\medskip

\end{figure}




\newpage
O conceito \texttt{LeiDeCota} é o elemento raiz  que pode ser instanciado no \gls{MPS} a fim de criar uma representação abstrata de uma nova lei de cotas na DSL. No \gls{MPS}, os conceitos raiz podem ser criados em módulos \texttt{Solutions}, os quais utilizam uma ou mais linguagens e são os responsáveis por conter o código fonte do usuário final (Figura \ref{fig:solutions}).

\begin{figure}[ht!]
\centering

\caption{\textmd{Criação de elementos raiz em  \texttt{Solutions} no \gls{MPS}}}
\label{fig:solutions}
\fcolorbox{gray}{white}{\includegraphics[width=\textwidth]{chapters/dslcotas/mps/imagens/solutions.png}}

\par\medskip\textbf{Fonte:} Elaborada pelo autor (2020). \par\medskip

\end{figure}



Por sua vez, uma \texttt{LeiDeCota} está associada aos elementos descritos na Tabela \ref{tblelementoslei}:

\begin{table}[ht]
\caption{Elementos associados ao conceito \texttt{LeiDeCota}}
\label{tblelementoslei}
\centering
\begin{tabular}{|p{4.2cm}|p{10cm}|}
\hline
\texttt{CodigoVersao}          & Elemento que mantém informações descritivas sobre a versão de lei aplicada.                                                                                           \\ \hline
Lista de \texttt{Configuracao} & Responsável por armazenar os parâmetros de configuração a serem reutilizados na DSL, contendo o nome da configuração e uma expressão de valor.                          \\ \hline
\texttt{Distribuicao}          & Conceito que contém a \texttt{CategoriaCota} raiz que dará início ao processo de distribuição das vagas entre as suas categorias filhas.                                       \\ \hline
\texttt{OrdemPrioridadeCotas}  & Elemento que contém a lista de referências para as categorias de cotas criadas durante a distribuição e será responsável por manter a ordem de prioridade prevista em lei. \\ \hline
\end{tabular}
  \par\medskip\textbf{Fonte:} Elaboração do autor (2020) \par\medskip
\end{table}

   
    

Para possibilitar a relação entre as regras definidas, alguns componentes da \gls{DSL} utilizam \texttt{references} para outros, possibilitando que elementos já definidos possam ser acessados pelos comandos \texttt{control+espaço} no \gls{MPS}. As \texttt{references} são restringidas pelo tipo do conceito alvo e pela cardinalidade, por exemplo, o conceito \texttt{CategoriaCotaRef} possui uma referência para uma \texttt{CategoriaCota} e, por sua vez, o elemento \texttt{OrdemPrioridadeCotas} possui lista de \texttt{CategoriaCotaRef} para que seja possível indicar na linguagem a ordem de prioridade criada durante a distribuição de vagas (Figura \ref{fig:references}).

\begin{figure}[ht!]
\centering

\caption{\textmd{Definição de \texttt{References} no \gls{MPS}}}
\label{fig:references}
\fcolorbox{gray}{white}{\includegraphics[width=\textwidth]{chapters/dslcotas/mps/imagens/references.png}}

\par\medskip\textbf{Fonte:} Elaborada pelo autor (2020). \par\medskip

\end{figure}



\newpage
Por fim, o conceito \texttt{Distribuicao} é o responsável por armazenar a árvore de distribuição, iniciando com a \texttt{CategoriaCota} raiz, na qual contém uma sigla, uma descrição, uma \texttt{Expression} onde será preenchida a reserva de vaga (percentuais fixos ou itens de \texttt{Configuracao} pré-definidos) e também uma lista de categorias filhas. 

Na Subseção \ref{sub:sec:editores}, serão apresentados os editores criados para definição da sintaxe de cada conceito definido na modelagem.


\subsection{\textit{Editores de conceitos}}
\label{sub:sec:editores}

\subsection{\textit{Restrições de escopo}}
\label{sub:sec:constraints}

\subsection{\textit{Comportamento dos elementos de conceito}}
\label{sub:sec:comportamentos}

\subsection{\textit{Sistema de tipos}}
\label{sub:sec:typesystem}

\subsection{\textit{Gerador textGen}}
\label{sub:sec:texgen}


  \section{A API DSL cotas}
\label{apicotas}

Em relação ao desafio abordado nos objetivos específicos dessa pesquisa, no que diz respeito à evolução entre as versões e à geração de código fonte de classificação e aprovação de candidatos, foi criada uma \gls{API} para prova de conceito sobre o uso dos padrões das regras definidas na DSL Cotas. Esta por sua vez implementa os algoritmos responsáveis por calcular o quadro de vagas, aprovar candidatos e definir a ordem de prioridade conforme as regras definidas pelo usuário.


Para tanto, a tecnologia escolhida foi o projeto \texttt{SpringBoot}, o qual segundo \citeonline{walls2016spring}, oferece um novo paradigma de desenvolvimento de aplicações com o \texttt{Spring Framework}, possibilitando desenvolver aplicativos com mais agilidade, focando em atender as necessidades de funcionalidade com o mínimo de configurações que for necessário.

No que concerne à escolha para criação de serviços web, o setor de desenvolvimento do \gls{IFSC} possui dois sistemas que envolvem processos seletivos, o primeiro é o sistema legado em \texttt{PHP}, tratado no Capítulo \ref{chap:historicoversoes} e o segundo é o novo Sistema Integrado de Gestão (SIG) que também contém um módulo responsável pela criação de processos seletivos e foi desenvolvido em \texttt{Java}. Portanto, a camada de serviços foi empregada com o intuito de permitir a utilização em sistemas distintos de modo independente de linguagem alvo para geração de código fonte.


Nas próximas Subseções são detalhados os recursos do \texttt{Spring} que foram utilizados na criação das funcionalidades de classificação e aprovação de candidatos.


\subsection{Componentes da API}
\label{componentesapi}

Inicialmente, foi gerado um projeto \texttt{SpringBoot} por meio do site \texttt{start.spring.io}, no qual foram marcadas as opções de módulos necessários para o desenvolvimento da pesquisa, tais como:

\begin{enumerate}
 
\item[a)] \texttt{Spring Boot Starter Web}: Principal dependência do \texttt{Spring}, que fornece a camada de desenvolvimento utilizada para construção de aplicações e de serviços web com o \texttt{spring-web}, incluindo um \texttt{tomcat} pré configurado e a biblioteca \texttt{jackson}, para fazer manipulação de (JSON ou XML);


\item[b)] \texttt{Spring Data JPA}:  Utilizado para simplificar a criação, seleção e manipulação das entidades de banco de dados criadas para processar o algoritmo de classificação e aprovação de candidatos; 

\item[c)] \texttt{JDBC API}: Para fornecer acesso ao banco de dados \texttt{MYSQL} do sistema de ingresso, utilizado para validação e possibilitar a comparação de resultados do histórico de candidatos dos processos do \gls{IFSC} com o resultado processado pela \gls{API} DSL Cotas; 


\item[d)] \texttt{Spring Reactive Web}: Módulo do \texttt{Spring} utilizado para testar as requisições para os \texttt{endpoints} implementados com base na DSL, por meio do \texttt{WebClient} em conjunto com o \texttt{JUnit}.

\end{enumerate}



Após a inicialização do projeto foram criadas as entidades de modelo necessárias para fazer o mapeamento do arquivo JSON gerado pela DSL Cotas, para isso, foram utilizadas as anotações \texttt{Json}, disponíveis pela biblioteca \texttt{jackson}. Um exemplo do mapeamento é apresentado pela Figura \ref{fig:jsonjackson}. 

\begin{figure}[ht!]
\centering

\caption{\textmd{Anotações com a biblioteca jackson}}
\label{fig:jsonjackson}
\fcolorbox{gray}{white}{\includegraphics[width=0.80\textwidth]{chapters/dslcotas/api/imagens/jsonjackson.png}}

\par\medskip\textbf{Fonte:} Elaborada pelo autor (2020). \par\medskip

\end{figure}



Desse modo, o modelo de regras gerado pela DSL Cotas é convertido para o objeto instância da classe \texttt{LeiDeCota}, que possui toda a estrutura de dados utilizada para armazenar e percorrer as regras definidas pelo usuário.

A API possui um controlador \texttt{REST} para cálculo de cotas, o qual fornece acesso às requisições \texttt{HTTP} que têm a função de: gerar o quadro de vagas, retornar a ordem de prioridade e aprovar uma lista de candidatos. Esse controlador por sua vez, utiliza os recursos do \texttt{Spring Data JPA} para conexão em um banco de dados H2, o qual é utilizado como meio de processar e aprovar a lista de candidatos passada.


O relacionamento entre os principais componentes da \gls{API} e a DSL Cotas pode ser observado na Figura \ref{fig:apicomponentes}. Adicionalmente, o Código Fonte \ref{lst:restcontroller} mostra as assinaturas das interfaces presentes no controlador \texttt{REST}.

\begin{figure}[ht!]
\centering

\caption{\textmd{Componentes da API DSL Cotas}}
\label{fig:apicomponentes}
\fcolorbox{gray}{white}{\includegraphics[width=0.88\textwidth]{chapters/dslcotas/api/imagens/apicomponentes.png}}

\par\medskip\textbf{Fonte:} Elaboração do autor (2020) \par\medskip

\end{figure}



\lstinputlisting[language=Java, 
caption=Endpoints no REST Controller  
,label=lst:restcontroller]{chapters/trechos_codigo/restcontroler.m}

\newpage
O método \texttt{quadro-vagas} (Código Fonte \ref{lst:restcontroller}, Linha 5), recebe como parâmetro a quantidade de vagas e o identificador da versão de lei, retornando uma lista de categorias com as respectivas siglas e quantidade de vagas (Figura \ref{fig:retquadrovagas}).

\begin{figure}[ht!]
\centering

\caption{\textmd{Retorno do método quadro-vagas}}
\label{fig:retquadrovagas}
\fcolorbox{gray}{white}{\includegraphics[width=0.98\textwidth]{chapters/dslcotas/api/imagens/retquadrovagas.png}}

\par\medskip\textbf{Fonte:} Elaborada pelo autor (2020). \par\medskip

\end{figure}



A aprovação é realizada no método \texttt{aprova-candidatos} (Código Fonte \ref{lst:restcontroller}, Linha 8), que recebe adicionalmente o parâmetro da lista de candidatos que deve ser processada. O seu retorno se dá por meio da mesma lista de candidatos, no entanto, já com a identificação da categoria de aprovação, conforme a versão da lei de cotas desejada (Figura \ref{fig:retaprovacandidatos}).
\begin{figure}[ht!]
\centering

\caption{\textmd{Retorno do método aprova-candidatos}}
\label{fig:retaprovacandidatos}
\fcolorbox{gray}{white}{\includegraphics[width=0.88\textwidth]{chapters/dslcotas/api/imagens/retaprovacaocandidatos.png}}

\par\medskip\textbf{Fonte:} Elaboração do autor (2020) \par\medskip

\end{figure}


A lista de candidatos é composta pela ordem de classificação, o código de inscrição, o código do curso a que o candidato concorre, a situação de inscrição, categoria de concorrência selecionada na inscrição e a situação de classificação. O campo \texttt{situacaoDeInscricao} inicia com a situação classificado (CLA), para que após o processamento somente os candidatos aprovados sejam marcados como aprovados (APV). A categoria de classificação conforme o sistema de cotas é retornada no campo \texttt{situacaoDeClassificacao}.

\newpage
A interface de ordem de prioridade é atribuída ao método \texttt{ordem-prioridade} (Código Fonte \ref{lst:restcontroller}, Linha 13), esse possui apenas o parâmetro identificador da lei e devolve a lista de siglas conforme a definição de prioridade feita pelo usuário na DSL Cotas (Figura \ref{fig:retordemprioridade}).

\begin{figure}[ht!]
\centering

\caption{\textmd{Retorno do método ordem-prioridade}}
\label{fig:retordemprioridade}
\fcolorbox{gray}{white}{\includegraphics[width=0.88\textwidth]{chapters/dslcotas/api/imagens/retordemprioridade.png}}

\par\medskip\textbf{Fonte:} Elaborada pelo autor (2020). \par\medskip

\end{figure}


\newpage
No Capítulo seguinte serão apresentados os procedimentos metodológicos utilizados na pesquisa.
 
