\chapter{Introdução}
\label{chap:intro}

Este capítulo contextualiza a justificativa e a relevância sobre o tema desta dissertação, assim como apresenta a organização do presente trabalho. 

Nas Seções \ref{contextualizacao} e \ref{motivacao}, são abordados os elementos contextuais e motivacionais para o desenvolvimento desse estudo, enquanto nas Seções \ref{problema} e \ref{objetivos}, são apontados o problema, as hipóteses e os objetivos da pesquisa. Por fim, a organização dos Capítulos é definida na Seção \ref{organizacao}.

 \section{Contextualização}
\label{contextualizacao}

 A constante alteração de regras de negócio advém de mudanças de lei, forças de mercado ou novos objetivos empresariais, exigindo que essas possam ser expressas de modo a serem compreendidas e facilmente alteradas pelas organizações, a fim de melhorar o desempenho de negócios \cite{flexiblerules}.
 
 Segundo \citeonline{verificationbusinessrules}, a mudança de regras de negócio, é inevitável, sendo cada vez mais frequente a adaptação às novas regulamentações e a necessidade de os sistemas estarem em conformidade com novas regras, que por décadas são implementadas em softwares de áreas industriais, financeiras, empresas de seguros, administração e várias outras.
 
 O \gls{IFSC} é uma instituição da rede de ensino pública federal presente no estado de Santa Catarina há mais de 100 anos, hoje desenvolve e mantém diversos sistemas de informação, os quais estão sujeitos a alterações para conformidade à legislação federal. As aplicações institucionais tem como principais envolvidos: discentes, professores e técnicos administrativos, os quais participam da oferta de cursos nos mais diversos níveis, tais como: cursos de qualificação profissional, educação jovens e adultos, técnicos, superiores e pós-graduação. 
 
 Ao que concerne à necessidade de adaptação a novas regras de negócio, o autor da presente pesquisa trata sobre um dos sistemas de informação mais utilizados internamente, o sistema de ingresso de discentes na instituição, o qual é mantido desde 2007 pela \gls{DTIC} e tem como objetivo disponibilizar ofertas de vagas em cursos por meio de processos seletivos. Atualmente, a base de dados do sistema consta com 788 processos seletivos e mais de 409 mil registros de candidatos.
 
 A principal funcionalidade do sistema de ingresso do \gls{IFSC} é a classificação de candidatos, a qual é construída e mantida conforme critérios da Lei nº 12.711/2012, que estabelece:
 \begin{citacao}
 Art. 1º As instituições federais de educação superior vinculadas ao Ministério da Educação reservarão, em cada concurso seletivo para ingresso nos cursos de graduação, por curso e turno, no mínimo 50\% (cinquenta por cento) de suas vagas para estudantes que tenham cursado integralmente o ensino médio em escolas públicas.
 
 Art. 3º Em cada instituição federal de ensino superior, as vagas de que trata o art. 1º desta Lei serão preenchidas, por curso e turno, por autodeclarados pretos, pardos e indígenas e por pessoas com deficiência, nos termos da legislação, em proporção ao total de vagas no mínimo igual à proporção respectiva de pretos, pardos, indígenas e pessoas com deficiência na população da unidade da Federação onde está instalada a instituição, segundo o último censo da Fundação Instituto Brasileiro de Geografia e Estatística - IBGE \cite{leicotas}.  
 \end{citacao}
 
 A adequação à legislação é demanda proveniente do \gls{MEC}, o qual faz referência em seu site aos decretos nº 7.824/2012, nº 9.034/2017 (que regulamentam a lei nº12.711) e às portarias normativas nº 18/2012 e nº 9/2017, as quais estabelecem regras, conceitos básicos e fórmulas para preenchimento das modalidades de reservas de vagas do sistema de cotas. 
 
 Sempre que o \gls{MEC} realiza alterações em um dos documentos de lei citados, é necessário também alterar os sistemas de informações das instituições de ensino. A exigência para adequação dos sistemas à lei traz o aumento da demanda de desenvolvimento, situação exemplificada no Capítulo  \ref{chap:historicoversoes}, no qual é elencado o histórico do controle de versão para 3 (três) versões já implementadas no \gls{IFSC}.
 
 

 
 

  \section{Motivação}
\label{motivacao}

O estudo de linguagens que sejam capazes de expressar regras de maneira mais clara não é novidade na área da engenharia de sistemas. \citeonline{bentley} já demonstrava em sua pesquisa diferentes abordagens de construção de pequenas linguagens para facilitar a escrita de programas gráficos e interface gráfica.


\citeonline{wexelblat}, apresenta o termo "linguagem de propósito especial", quando se refere à \gls{JOVIAL}, linguagem de alto nível destinada principalmente para auxiliar na programação de grandes sistemas complexos em tempo real. 

Um exemplo de aplicação dessas linguagens pode ser visto na Figura \ref{fig:piclanguage} em que o compilador transforma comandos simples no formato textual em formato de digrama (Figura \ref{fig:piclanguageresultado}), abstraindo detalhes específicos das linguagens tradicionais de programação da época.

\begin{figure}[ht!]
\centering

\caption{\textmd{Comandos na PIC Language}}
\label{fig:piclanguage}
\fcolorbox{gray}{white}{\includegraphics[width=0.67\textwidth]{images/piclanguage}}

\par\medskip\textbf{Fonte:} \citeonline{bentley}. \par\medskip
\end{figure}

\begin{figure}[ht!]
\centering

\caption{\textmd{Diagrama resultado após compilação}}
\label{fig:piclanguageresultado}
\fcolorbox{gray}{white}{\includegraphics[width=0.67\textwidth]{images/piclanguageresultado}}

\par\medskip\textbf{Fonte:} \citeonline{bentley}. \par\medskip
\end{figure}



Na engenharia de software atual, as Linguagens de Domínio Específico ou DSLs, do inglês \textit{Domain Specific Languages}, estão se tornando cada vez mais importantes e as novas ferramentas de criação dessas linguagens tem evoluído, reduzindo o esforço de desenvolvimento \cite{dslengineering}.


A equipe de desenvolvedores do \gls{IFSC} possui uma alta demanda de desenvolvimento de sistemas e serviços internos, atualmente são mantidos mais de 10 sistemas em uso, alguns deles subdivididos em vários módulos \cite{catalogoifsc}. A \gls{DTIC} centraliza e mantém a maioria dos sistemas e serviços de TI, contando com apenas 12 analistas da tecnologia da informação, além de prestar suporte para todas as 23 unidades da instituição. 

Como agravante, o sistema de ingresso foco deste trabalho, foi criado em meados dos anos 2000 por bolsistas que já não estão mais na instituição, e atualmente apenas 2 desenvolvedores são responsáveis pelas demandas de desenvolvimento e suporte. Por se tratar de um sistema legado criado em linguagem PHP sem qualquer preocupação com documentação ou qualidade de código, o custo de alterações mais complexas como no caso de regras de classificação acaba por atrasar a adequação aos novos requisitos de lei. 

Até o momento, a presente equipe participou do desenvolvimento de 3 (três) versões do algoritmo de classificação, além de prestar manutenção corretiva em função de entendimentos equivocados nas regras implementadas.  Os algoritmos envolvidos, assim como o seu histórico de versionamento são detalhados no Capítulo \ref{chap:historicoversoes}.

O Capítulo \ref{chap:proposta} apresenta a proposta de uma Linguagem de Domínio Específico para especificação de regras de classificação de candidatos, na qual usuários especialistas do sistema de ingresso possam estabelecer as categorias de cotas, com objetivo de reduzir o esforço do desenvolvimento e de entendimento das regras de negócio, por parte dos desenvolvedores do sistema, sempre que forem alterados os documentos de lei envolvidos.

Nas Seções \ref{problema}, \ref{objetivos}  e \ref{metodologia}, são apresentados, respectivamente, o problema de pesquisa, as hipóteses, os objetivos e por fim a metodologia de pesquisa utilizada.

  
  \section{Problema de pesquisa e Hipótese}
\label{problema}

Essa pesquisa surge em decorrência das necessidades recentes para refatoração do código fonte do sistema de ingresso do \gls{IFSC}. Essas necessidades foram resultadas em função de alterações nos documentos de lei e em função de  divergências sobre o entendimento de distribuição de vagas entre desenvolvedores e especialistas de negócio. Essas questões acabam atrasando o processo de implementação do sistema para aderência a uma nova legislação ou a uma instrução normativa definida pelos órgãos de controle.

Do mesmo modo, essas situações trazem alguns desafios para a equipe de desenvolvedores e para os \textit{stakeholders} que analisam a legislação e definem os requisitos do sistema de convocação de candidatos cotistas nos cursos do \gls{IFSC}. 

No que concerne ao apoio ao desenvolvimento e especificação desse problema de pesquisa, podem ser listados os seguintes questionamentos:

\begin{itemize}
    \item Como dar liberdade aos usuários para definir as regras do sistema de cotas, e utilizar essas definições de modo a serem implantadas diminuindo a dependência da equipe de  desenvolvedores?
    
    \item É possível a criação de uma linguagem específica de domínio que padronize as definições em comum nas regras de distribuição de vagas?
    
    \item A utilização da linguagem proposta pode reduzir a quantidade de linhas de código a serem implementadas manualmente pelos desenvolvedores?
    
    \item Quais são as ferramentas utilizadas para construção de \gls{DSL}s e de que modo podem permitir a geração de código fonte para implementação do algoritmo de classificação por cotas?
    
    \item A definição de regras do sistema de cotas pode ser facilitada com apoio de uma \gls{IDE} que faça validações e forneça recursos específicos para a linguagem proposta?
    

    
\end{itemize}{}
 
 \section{Objetivos}
\label{objetivos}

\subsection{Objetivo Geral}
\label{objetivogeral}

Compreender, por meio da elaboração de uma linguagem específica de domínio, a viabilidade de melhoria na comunicação entre usuários de negócio e desenvolvedores, visando o aumento na produtividade da especificação de requisitos e da implementação de regras concernentes ao sistema de cotas da rede de ensino pública federal. 


\subsection{Objetivos Específicos}
\label{objetivosespecificos}

\begin{enumerate}
    \item[a)] Realizar levantamento do histórico de versões presentes no sistema de controle de versão do \gls{IFSC} sobre regras de classificação de cotas;
    \item[b)] Analisar e identificar as características em comum entre as versões do algoritmo implementadas para que seja possível elaborar a modelagem da \gls{DSL} proposta;
    \item[c)] Definir e implementar uma \gls{DSL} para usuários especialistas nas regras do sistema de cotas;
    \item[d)] Criar um cenário de avaliação da DSL, a fim de validar o seu entendimento por usuários de negócio e desenvolvedores;
    \item[e)] Implementar uma \gls{API} utilizando as regras definidas pelos usuários de negócio para geração do algoritmo de classificação pelo sistema de cotas.
\end{enumerate}{}
      
 \section{Organização do trabalho}
\label{organizacao}

O presente trabalho foi organizado nos seguintes capítulos:

\begin{itemize}
    \item Capítulo \ref{chap:intro}: Introdução - que apresenta a motivação, a contextualização, os objetivos e a organização do trabalho;
    \item Capítulo \ref{chap:historicoversoes}: Sistema de Ingresso e Versionamento - que elenca o histórico de mudanças no sistema de cotas do sistema de ingresso do \gls{IFSC}, o que motivou o estudo inicial desta pesquisa;
    \item Capítulo \ref{metodologia}: Metodologia - no qual foi descrita a classificação e as etapas da pesquisa, o ambiente da pesquisa e os métodos de avaliação da DSL;
    \item Capítulo \ref{chap:fundamentacao}: Fundamentação Teórica - o qual abordou a revisão da literatura sobre os temas ligados ao desenvolvimento de \gls{DSL} na engenharia de software, as diferenças sobre \gls{GPL}, as vantagens e desvantagens do uso  \gls{DSL}, assim como as ferramentas utilizadas para a sua construção;
    \item Capítulo \ref{chap:dslcotas}: DSL de Cotas - que abordou uma \gls{DSL} como meio de simplificar a especificação e desenvolvimento de regras de classificação de candidatos ao sistema de cotas da rede de ensino pública federal, assim como apresentou a \gls{API} DSL Cotas, responsável por implementar o serviço de classificação e aprovação de candidatos;
    \item Capítulo \ref{chap:analise}: Avaliação e Análise dos Resultados - o qual contém a análise dos dados coletados por meio de um exercício prático de avaliação da DSL Cotas, de um questionário aplicado com os usuários e da avaliação da
    \gls{API} DSL Cotas;
    \item Capítulo \ref{chap:consideracoes}: Conclusões - na qual foi realizada uma síntese dos principais resultados da pesquisa, bem como os trabalhos correlatos, o espoco negativo da pesquisa, as principais contribuições da DSL Cotas, os trabalhos futuros sugeridos e as considerações finais.
\end{itemize}


 
% \section{Organização do trabalho}
\label{organizacao}

O presente trabalho foi organizado nos seguintes capítulos:

\begin{itemize}
    \item Capítulo \ref{chap:intro}: Introdução - que apresenta a motivação, a contextualização, os objetivos e a organização do trabalho;
    \item Capítulo \ref{chap:historicoversoes}: Sistema de Ingresso e Versionamento - que elenca o histórico de mudanças no sistema de cotas do sistema de ingresso do \gls{IFSC}, o que motivou o estudo inicial desta pesquisa;
    \item Capítulo \ref{metodologia}: Metodologia - no qual foi descrita a classificação e as etapas da pesquisa, o ambiente da pesquisa e os métodos de avaliação da DSL;
    \item Capítulo \ref{chap:fundamentacao}: Fundamentação Teórica - o qual abordou a revisão da literatura sobre os temas ligados ao desenvolvimento de \gls{DSL} na engenharia de software, as diferenças sobre \gls{GPL}, as vantagens e desvantagens do uso  \gls{DSL}, assim como as ferramentas utilizadas para a sua construção;
    \item Capítulo \ref{chap:dslcotas}: DSL de Cotas - que abordou uma \gls{DSL} como meio de simplificar a especificação e desenvolvimento de regras de classificação de candidatos ao sistema de cotas da rede de ensino pública federal, assim como apresentou a \gls{API} DSL Cotas, responsável por implementar o serviço de classificação e aprovação de candidatos;
    \item Capítulo \ref{chap:analise}: Avaliação e Análise dos Resultados - o qual contém a análise dos dados coletados por meio de um exercício prático de avaliação da DSL Cotas, de um questionário aplicado com os usuários e da avaliação da
    \gls{API} DSL Cotas;
    \item Capítulo \ref{chap:consideracoes}: Conclusões - na qual foi realizada uma síntese dos principais resultados da pesquisa, bem como os trabalhos correlatos, o espoco negativo da pesquisa, as principais contribuições da DSL Cotas, os trabalhos futuros sugeridos e as considerações finais.
\end{itemize}

 
% \input{figures/densenet}

%\input{equations/ce}

%\input{tables/deep_datasets}

% \input{algorithms/algorithm} 
