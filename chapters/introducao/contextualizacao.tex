\section{Contextualização}
\label{contextualizacao}

 Para \citeonline{ghosh2011dsl}, problemas na comunicação entre desenvolvedores e especialistas de domínio são os principais motivos de falha no desenvolvimento e evolução de projetos de software. Especialistas, entendem a terminologia do domínio e falam em um vocabulário que pode ser estranho para as equipes de desenvolvedores. Nesse sentido, é preciso identificar meios para redução da lacuna semântica entre especialistas e desenvolvedores, de modo que os especialistas possam estar envolvidos na verificação de regras de negócio ao longo do ciclo de vida do projeto.

 A constante alteração de regras de negócio advém de mudanças de lei, forças de mercado ou novos objetivos empresariais, exigindo que essas possam ser expressas de modo a serem compreendidas e facilmente alteradas pelas organizações, a fim de melhorar o desempenho de negócios \cite{flexiblerules}.
 
 Segundo \citeonline{verificationbusinessrules}, a mudança de regras de negócio é inevitável, sendo cada vez mais frequente a adaptação às novas regulamentações e a necessidade de os sistemas estarem em conformidade com novas regras, que por décadas são implementadas em softwares de áreas industriais, financeiras, empresas de seguros, administração e várias outras.
 
 O \gls{IFSC} é uma instituição da rede de ensino pública federal presente no estado de Santa Catarina há mais de 100 anos, que hoje desenvolve e mantém diversos sistemas de informação, os quais estão sujeitos a alterações para conformidade à legislação federal. As aplicações institucionais tem como principais envolvidos os discentes, professores e técnicos administrativos, os quais participam da oferta de cursos nos mais diversos níveis, tais como: cursos de qualificação profissional, educação jovens e adultos, técnicos, superiores e pós-graduação. 
 
 Ao que concerne à necessidade de adaptação a novas regras de negócio, o autor da presente pesquisa trata sobre um dos sistemas de informação mais utilizados internamente, o sistema de ingresso de discentes na instituição, o qual é mantido desde 2007 pela \gls{DTIC} e tem como objetivo disponibilizar ofertas de vagas em cursos por meio de processos seletivos. Atualmente, a base de dados do sistema conta com 788 processos seletivos e mais de 409 mil registros de candidatos.
 
 A principal funcionalidade do sistema de ingresso do \gls{IFSC} é a classificação de candidatos, a qual é construída e mantida conforme critérios da Lei nº 12.711/2012, que estabelece:
 \begin{citacao}
 Art. 1º As instituições federais de educação superior vinculadas ao Ministério da Educação reservarão, em cada concurso seletivo para ingresso nos cursos de graduação, por curso e turno, no mínimo 50\% (cinquenta por cento) de suas vagas para estudantes que tenham cursado integralmente o ensino médio em escolas públicas.
 
 Art. 3º Em cada instituição federal de ensino superior, as vagas de que trata o art. 1º desta Lei serão preenchidas, por curso e turno, por autodeclarados pretos, pardos e indígenas e por pessoas com deficiência, nos termos da legislação, em proporção ao total de vagas no mínimo igual à proporção respectiva de pretos, pardos, indígenas e pessoas com deficiência na população da unidade da Federação onde está instalada a instituição, segundo o último censo da Fundação Instituto Brasileiro de Geografia e Estatística - IBGE \cite{leicotas}.  
 \end{citacao}
 
 A adequação à legislação é demanda proveniente do \gls{MEC}, o qual faz referência em seu site aos decretos nº 7.824/2012, nº 9.034/2017 (que regulamentam a lei nº12.711) e às portarias normativas nº 18/2012 e nº 9/2017, as quais estabelecem regras, conceitos básicos e fórmulas para preenchimento das modalidades de reservas de vagas do sistema de cotas. 
 
 Sempre que o \gls{MEC} realiza alterações em um dos documentos de lei citados, é necessário também alterar os sistemas de informações das instituições de ensino. A exigência para adequação dos sistemas à lei traz o aumento da demanda de desenvolvimento, situação ilustrada no Capítulo  \ref{chap:historicoversoes}, no qual é elencado o histórico do controle de versão para 3 (três) versões já implementadas no \gls{IFSC}.
 
 

 
 