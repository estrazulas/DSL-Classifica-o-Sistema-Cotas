\section{Organização do trabalho}
\label{organizacao}

O presente trabalho é organizado nos seguintes capítulos:

\begin{itemize}
    \item Capítulo \ref{chap:intro}: Introdução - que apresenta a motivação, a contextualização, os objetivos e a organização do trabalho;
    \item Capítulo \ref{chap:historicoversoes}: Sistema de Ingresso e Versionamento - que elenca o histórico de mudanças no sistema de cotas do sistema de ingresso do \gls{IFSC}, o que motivou o estudo inicial desta pesquisa;
    \item Capítulo \ref{metodologia}: Metodologia - no qual é descrita a classificação e as etapas da pesquisa, o ambiente da pesquisa e os métodos de avaliação da DSL;
    \item Capítulo \ref{chap:fundamentacao}: Fundamentação Teórica - o qual aborda a revisão da literatura sobre os temas ligados ao desenvolvimento de \gls{DSL} na engenharia de software, as diferenças sobre \gls{GPL}, as vantagens e desvantagens do uso  \gls{DSL}, assim como as ferramentas utilizadas para a sua construção;
    \item Capítulo \ref{chap:dslcotas}: DSL de Cotas - que aborda uma \gls{DSL} como meio de simplificar a especificação e desenvolvimento de regras de classificação de candidatos ao sistema de cotas da rede de ensino pública federal, assim como apresenta a \gls{API} DSL Cotas, responsável por implementar o serviço de classificação e aprovação de candidatos;
    \item Capítulo \ref{chap:analise}: Avaliação e Análise dos Resultados - o qual contém a análise dos dados coletados por meio de um exercício prático de avaliação da DSL Cotas, de um questionário aplicado com os usuários e da avaliação da
    \gls{API} DSL Cotas;
    \item Capítulo \ref{chap:consideracoes}: Conclusões - na qual é realizada uma síntese dos principais resultados da pesquisa, bem como os trabalhos relacionados, o espoco negativo da pesquisa, as principais contribuições da DSL Cotas, os trabalhos futuros sugeridos e as considerações finais.
\end{itemize}
