\section{Organização do trabalho}
\label{organizacao}

O presente trabalho está organizado nos seguintes capítulos:

\begin{itemize}
    \item Capítulo \ref{chap:intro}: Introdução - que apresenta a motivação, contextualização, objetivos e metodologia utilizada;
    \item Capítulo \ref{chap:historicoversoes}: Sistema de Ingresso e Versionamento - elencando o histórico de mudanças no sistema de cotas do sistema de ingresso do \gls{IFSC} o que motivou o estudo inicial desta pesquisa;
    \item Capítulo \ref{chap:fundamentacao}: Fundamentação Teórica - abordando a revisão da literatura sobre temas ligados ao desenvolvimento de \gls{DSL} na engenharia de software, as diferenças sobre \gls{GPL}, as vantagens e desvantagens do uso  \gls{DSL}, assim como as ferramentas utilizadas para sua construção;
    \item Capítulo \ref{chap:dslcotas}: DSL de Cotas - abordando uma \gls{DSL} como meio de simplificar a especificação e desenvolvimento de regras de classificação de candidatos ao sistema de cotas da rede de ensino pública federal, assim como apresentar a \gls{API} DSL Cotas, responsável por implementar o serviço de classificação e aprovação de candidatos;
    \item Capítulo \ref{chap:analise}: Avaliação e Análise dos Resultados - contendo a apresentação e analise dos dados coletados, por meio do exercício prático de avaliação da DSL Cotas, de um questionário aplicado com os usuários e da avaliação da
    \gls{API} DSL Cotas;
     \item Capítulo \ref{chap:trabalhoscorrelatos}: Trabalhos Correlatos - abordando outras \gls{DSL}s que fazem abstração de regras de negócio, envolvendo cálculos e processamento de questões legais;  
    \item Capítulo \ref{chap:consideracoes}: Considerações Finais - contendo uma síntese dos principais resultados da pesquisa, bem como as principais contribuições da DSL Cotas e trabalhos futuros.
\end{itemize}
