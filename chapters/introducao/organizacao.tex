\section{Organização do trabalho}
\label{organizacao}

O presente trabalho será organizado nos seguintes capítulos:

\begin{itemize}
    \item Capítulo \ref{chap:intro}: Introdução - que apresenta a motivação, contextualização, objetivos e metodologia utilizada;
    \item Capítulo \ref{chap:historicoversoes}: Sistema de Ingresso e Versionamento - elencando o histórico de mudanças no sistema de cotas do sistema de ingresso do \gls{IFSC} o que motivou o estudo inicial desta pesquisa;
    \item Capítulo \ref{chap:fundamentacao}: Fundamentação Teórica - abordando a revisão da literatura sobre temas ligados ao desenvolvimento de \gls{DSL} na engenharia de software, as diferenças sobre \gls{GPL}, as vantagens e desvantagens do uso  \gls{DSL}, assim como as ferramentas utilizadas para sua construção;
    \item Capítulo \ref{chap:proposta}: Proposta - abordando uma \gls{DSL} como meio de simplificar a especificação e desenvolvimento de regras de classificação de candidatos ao sistema de cotas da rede de ensino pública federal;
    \item Capítulo \ref{chap:consideracoes}: Considerações Finais - com a apresentação do cronograma e os resultados esperados para esta pesquisa.
\end{itemize}
