\section{Problema de pesquisa e Hipótese}
\label{problema}

Essa pesquisa surge em decorrência das necessidades recentes para refatoração do código fonte do sistema de ingresso do \gls{IFSC}. Essas necessidades foram resultadas em função de alterações nos documentos de lei e em função de  divergências sobre o entendimento de distribuição de vagas entre desenvolvedores e especialistas de negócio. Essas questões acabam atrasando o processo de implementação do sistema para aderência a uma nova legislação ou a uma instrução normativa definida pelos órgãos de controle.

Do mesmo modo, essas situações trazem alguns desafios para a equipe de desenvolvedores e para os \textit{stakeholders} que analisam a legislação e definem os requisitos do sistema de convocação de candidatos cotistas nos cursos do \gls{IFSC}. 

No que concerne ao apoio ao desenvolvimento e especificação desse problema de pesquisa, podem ser listados os seguintes questionamentos:

\begin{itemize}
    \item Como dar liberdade aos usuários para definir as regras do sistema de cotas, e utilizar essas definições de modo a serem implantadas diminuindo a dependência da equipe de  desenvolvedores?
    
    \item É possível a criação de uma linguagem específica de domínio que padronize as definições em comum nas regras de distribuição de vagas?
    
    \item A utilização da linguagem proposta pode reduzir a quantidade de linhas de código a serem implementadas manualmente pelos desenvolvedores?
    
    \item Quais são as ferramentas utilizadas para construção de \gls{DSL}s e de que modo podem permitir a geração de código fonte para implementação do algoritmo de classificação por cotas?
    
    \item A definição de regras do sistema de cotas pode ser facilitada com apoio de uma \gls{IDE} que faça validações e forneça recursos específicos para a linguagem proposta?
    

    
\end{itemize}{}