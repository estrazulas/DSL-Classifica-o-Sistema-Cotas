\section{Problema de pesquisa e Hipótese}
\label{problema}

Essa pesquisa surge em decorrência das necessidades recentes para refatoração do código fonte do sistema de ingresso do \gls{IFSC}. Essas necessidades foram resultadas em função de alterações nos documentos de lei e em função de  divergências sobre o entendimento de distribuição de vagas entre desenvolvedores e especialistas de negócio. Essas questões acabam atrasando o processo de implementação do sistema para aderência à uma nova legislação ou à uma instrução normativa definida pelos órgãos de controle.

Do mesmo modo, essas situações trazem alguns desafios para a equipe de desenvolvedores e para os \textit{stakeholders} que analisam a legislação e definem os requisitos do sistema de convocação de candidatos cotistas nos cursos do \gls{IFSC}. 

No que concerne ao problema de pesquisa destaca-se: 
Como utilizar uma linguagem de definição de regras do sistema de cotas de modo a melhorar a comunicação entre usuários especialistas e desenvolvedores e, assim, verificar a possibilidade de melhoria no processo de desenvolvimento das regras presentes na legislação?

Assim, a partir da pergunta central, podem ser listados os seguintes questionamentos:

\begin{enumerate}
    
    \item[a)] É possível a criação de uma linguagem específica de domínio que padronize as definições em comum nas regras de distribuição de vagas?
    
    \item[b)] A utilização da linguagem proposta pode ajudar na produtividade durante o desenvolvimento de alterações na legislação, reduzindo a quantidade de linhas de código a serem implementadas manualmente pelos desenvolvedores?
    
    \item[c)] A definição de regras do sistema de cotas pode ser facilitada com apoio de uma \gls{IDE} que faça validações e forneça recursos específicos para a linguagem proposta?
    

    
\end{enumerate}