\chapter{Sistema de Ingresso e Versionamento}
\label{historicoversoes}

Neste capítulo serão descritas informações sobre as funcionalidades desenvolvidas no sistema de ingresso do \gls{IFSC} com relação aos requisitos e algoritmos do sistema de cotas, para este fim será utilizado o histórico do controle de versão no que concerne ao quantitativo de arquivos, classes, funções/métodos e as diferentes versões desde o surgimento da demanda de cotas na legislação.

\section{Histórico de Projeto}
\label{historicopj}
Criado em meados do ano de 2000 o sistema tem por objetivo disponibilizar vagas de cursos para os discentes do \gls{IFSC}. Este sistema foi desenvolvido internamente na linguagem PHP, para automatizar os processos seletivos, que eram realizados por meio de planilhas e ferramentas não integradas, as quais demandavam ao setor responsável muitas pessoas e muitos procedimentos operacionais repetitivos, gerando falhas no processo por erro humano.

O projeto não utiliza conceitos de orientação a objetos, em sua maioria os arquivos PHP ultrapassam duas mil linhas, sem divisão em camadas \gls{MVC}, com combinações das linguagens Javascript, HTML e PHP no mesmo arquivo. Quando era preciso criar ou adaptar alguma nova funcionalidade, por falta de conhecimento técnico os antigos desenvolvedores (bolsistas) faziam a cópia das funcionalidades para vários locais do sistema, sem pensar em reutilização de código.

Com objetivo de elencar a situação atual do código fonte do sistema, neste trabalho serão apresentados os quantitativos levantados a partir do sistema de controle de versão do \gls{IFSC}. O Quadro \ref{} apresenta o levantamento geral sobre o total de arquivos, linhas de código, commits e desenvolvedores que já atuaram no projeto. Nas seções seguintes serão contextualizadas as 3 (três) versões do algoritmo de classificação já implementadas em função da legislação.

\section{Algoritmo de Classificação}
\label{algoritimodeclassificacao}

Neste seção serão detalhados os conceitos e as etapas do processo de classificação de candidatos à vagas do sistema de cotas, assim como o levantamento de cenários exemplifica-tórios sobre distribuição de vagas tendo como base as versões já implementadas no sistema de ingresso.

Para cada processo seletivo há uma lista de cursos, onde o setor que gerencia as vagas define o valor total de vagas iniciais e indica se o processo seletivo vai utilizar a regra de classificação por cotas. Durante as inscrições serão apresentados aos candidatos os campos necessários para indicar que vieram de escola pública e também informar sua renda familiar e se autodeclaram pretos, pardos e indígenas.

Após o término do período de inscrições os candidatos são separados em categorias de concorrência de acordo com o preenchimento realizado no ato da inscrição. A seguir, os candidatos participam do processo seletivo de forma física ( provas de vestibular ) ou eletrônica ( sorteio, pontuação por preenchimento de formulários sócio-econômicos ou classificação por ENEM/SISU ). Por fim é gerado um número de classificação que representa a ordem dos candidatos que disputam as vagas disponíveis.

O algoritmo de classificação utiliza como parâmetro de entrada a lista de candidatos, contendo o número de inscrição, a ordem de classificação geral, a data de nascimento (como critério de desempate), a quantidade de vagas total do curso, o percentual de vagas disponível para escola pública e o percentual de proporção do \gls{IBGE}, que varia conforme a Unidade Federada e é fornecido conforme o ultimo censo demográfico, representando o percentual sobre o total de pretos, pardos e indígenas no estado em relação às demais categorias.

Com esses parâmetros de entrada o algoritmo gera um quadro de vagas contendo quantas vagas estão reservadas para às respectivas categorias de cotas, e faz a seleção e aprovação de candidatos de acordo com a sua classificação e critérios de desempate. Por fim, em caso de sobra de vagas por falta de candidatos para uma determinada categoria de cota, o algoritmo utiliza faz nova busca por candidatos de outra categoria conforme ordem de prioridade estabelecida na portaria Nº 18 de 2012 do \gls{MEC}, a qual define:

\begin{citacao}
Art. 15. No caso de não preenchimento das vagas reservadas aos autodeclarados pretos, pardos
e indígenas e às pessoas com deficiência, aquelas remanescentes serão preenchidas pelos
estudantes que tenham cursado integralmente o ensino fundamental ou médio, conforme o caso,
em escolas públicas, observadas as reservas realizadas em mesmo nível ou no imediatamente
anterior, nos termos do art. 10 desta Portaria. \cite{portarianr9}
\end{citacao}

Ao final do processo de classificação é gerada uma lista de candidatos aprovados, incluindo a classificação geral e a classificação na respectiva categoria de cota, por fim esta lista é enviada ao sistema acadêmico para que os candidatos possam realizar a matrícula e entregar a documentação necessária. 

As alterações nos documentos de lei podem incluir novas categorias, novas formas de distribuição das vagas iniciais, mudanças de percentuais, formas de arredondamento, descrição dos tipos de cotas e mudança na ordem de prioridade em caso de sobra de vaga. Deste modo, nas seções seguintes serão relatados os casos e cenários onde no \gls{IFSC} houve maior impacto de refatoração de código para adequação à legislação.


\subsection{Versão 1 - Início do sistema de Cotas, Lei Nº 12.711 de 2012}
\label{versao1}

\subsection{Versão 2 - Alteração para Lei Nº 13.409 de 2016 }
\label{versao1}

\subsection{Versão 3 - Reimplementação para interpretação do MEC em 2017 }
\label{versao1}

\subsection{Outras customizações realizadas no algoritmo}
\label{versao1}