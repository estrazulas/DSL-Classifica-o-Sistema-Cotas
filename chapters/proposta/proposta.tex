\chapter{Proposta}
\label{chap:proposta}

   Nesse Capítulo, é descrita a proposta de linguagem a ser desenvolvida na ferramenta \gls{MPS}, que possui como principal objetivo facilitar a definição das regras de distribuição de cotas e, dessa forma, permitir que os usuários do sistema de ingresso do \gls{IFSC} possam alterar essas definições de forma independente à implementação do respectivo algoritmo de classificação.
   
   Como foi descrito no Capítulo \ref{chap:historicoversoes}, desde a promulgação da lei de cotas em 2012, o sistema de ingresso passou por, pelo menos, 3 (três) versões diferentes de distribuição de vagas, ocasionando em várias refatorações de código desde então. 
   
   Desse modo, a presente pesquisa tem como proposta: a definição de uma linguagem que permita descrever todos os 3 (três) cenários passados, com apoio de recursos de linguagens existentes no \gls{MPS} tais como: validação de estrutura e de tipos de dados aceitos, preenchimento automático de código de acordo com o contexto, validação de inconsistências, recursos de dicas ao usuário da linguagem e geração de código fonte por meio de transformações. 
   
   Com apoio desses recursos, o usuário especialista no sistema de cotas, poderá fazer o uso controlado da especificação de regras, assim como, no caso de surgimento de um novo cenário de distribuição de vagas. Esse conjunto de regras alteradas poderiam ser transformadas em código fonte, possibilitando a  sua importação no sistema de ingresso do \gls{IFSC}, diminuindo a necessidade de refatoração em seu código.
   
   Nesse sentido, a Seção \ref{sec:dslproposta:usuario} aborda a proposta de uma \gls{DSL} a ser utilizada por especialistas de negócio, para definição da árvore de distribuição de cotas, tendo como base as especificações de lei. Enquanto na Seção \ref{sec:dslproposta:dev}, é descrita outra proposta de \gls{DSL}, que utiliza a primeira como dependência, e se aplica a desenvolvedores, porém o foco da modelagem trata de questões técnicas sobre a geração de código do algoritmo de aprovação.
   
   A segunda \gls{DSL}, tem como proposta definir: configurações relacionadas à forma de entrada de dados da lista de candidatos, assinatura do método a ser gerado, linguagem alvo da geração, formato da lista de retorno dos aprovados e a composição de campos para cada uma das respectivas categorias de cotas configuradas pelos usuários na primeira \gls{DSL}.
   
   \section{DSL de distribuição de vagas}
\label{sec:dslproposta:usuario}

   
 Tendo como base a análise realizada no Capítulo \ref{chap:historicoversoes}, nas Seções \ref{versao1}, \ref{versao2} e \ref{versao3}, são pontuadas algumas características em comum identificadas entre cada uma dessas versões:
   
   \begin{enumerate}
    \item[a)] A divisão das vagas entre as diferentes categorias de cotas, se deu em formato hierárquico, iniciando no total de vagas o qual foi sendo subdividido em percentuais reservados para suas subcategorias (ex: Estudantes da escola pública, Estudantes com renda baixa, Estudantes PCD, etc); 
   
   \item[b)] Esses percentuais são de conhecimento dos usuários especialistas no sistema de cotas, e podem variar de acordo com a documentação da legislação vigente, ou outras definições que variam conforme a unidade federada que oferta o curso; 
   
   \item[c)] Em muitos casos, não é definido um valor de percentual fixo, de modo que uma categoria recebe o valor calculado restante de vagas da categoria pai. Por exemplo, o percentual de cotistas \gls{PCD} é aplicado, e o que resta das vagas vai para candidatos da mesma categoria que não são \gls{PCD};

   \item[d)] Todas as versões consideravam a ordem de prioridade entre as diferentes categorias de cotas passíveis de inscrição;
   
   \item[e)] Questões de arredondamento das vagas (para cima ou para baixo), devem ser consideradas e podem variar de acordo com a versão de lei implementada.

   \end{enumerate}
   
   Essa \gls{DSL} proposta, objetiva abordar um modelo que permita a configuração de todos os itens listados, tais como: identificador da versão de lei, lista de variáveis para configuração de percentuais, formas de aplicação de arredondamento, uma macro para aplicar a função de resto de vagas de uma categoria, campo descritivo para os diferentes tipos de categorias, estrutura para divisão em subcategorias e lista de ordem de prioridade para sobra de vagas.
   
   No Código Fonte \ref{lst:dsl1versaoconfig}, é apresentado um exemplo de instância da configuração de distribuição de vagas, em que a sintaxe requer uma string para definição da versão, e uma lista de variáveis a serem criadas pelo usuário da linguagem para que sejam definidos os percentuais de reserva de vagas.
   
   \lstinputlisting[language=PHP, 
caption=Configurações de percentuais
,label=lst:dsl1versaoconfig]{chapters/proposta/codigos_proposta/versao_config_proposta.m}


No que diz respeito à estrutura hierárquica, no Código Fonte \ref{lst:dsl1arvorecotas} é possível identificar os campos da árvore de distribuição ("Distribuição de vagas:"), iniciando com o identificador referente à sigla da cota, e suas respectivas propriedades, tais como: descrição da categoria de cota, o percentual de reserva a ser aplicado (variável configurada, valor percentual preenchido manualmente ou a macro "Restante das vagas") e, por fim, se haverá novas subdivisões em outras categorias ("Se divide em:").

   \lstinputlisting[language=PHP, 
caption=Definição da estrutura de distribuição de vagas
,label=lst:dsl1arvorecotas]{chapters/proposta/codigos_proposta/arvore_cotas.m}


O último ponto a ser exemplificado é o caso da configuração da ordem de prioridade ("Ordem prioritária de preenchimento para sobra de vagas:") em caso de sobra de vagas (Código Fonte \ref{lst:dsl1prioridade}), o usuário poderá selecionar as categorias preenchidas anteriormente com a marcação "Quadro de vagas:" de valor "Sim".  

   \lstinputlisting[language=PHP, 
caption=Ordem de preenchimento para sobra de vagas
,label=lst:dsl1prioridade]{chapters/proposta/codigos_proposta/prioridade.m}

É importante ressaltar que, com apoio do \gls{MPS} é possível utilizar recursos de preenchimento automático de código, avisar o usuário da linguagem sobre inconsistências como: declarações duplicadas, formatos inválidos de entradas de dados, categorias de cotas com configurações incompletas ou não preenchidas e ausência de categorias na lista de ordem de prioridade (preenchimento em caso de sobra de vagas). 

Na Seção \ref{sec:dslproposta:dev}, tratamos outros pontos sobre a modelagem para selecionar candidatos em banco de dados, ou em outro tipo de estrutura de dados. Nesse sentido, a proposta é separar questões mais técnicas em uma linguagem à parte, de modo que a \gls{DSL} descrita anteriormente, seja utilizada como dependência, e seja feito o mapeamento com informações para geração de código fonte em uma linguagem alvo.

   \section{DSL técnica para geração algoritmo de aprovação}
\label{sec:dslproposta:dev}

 Uma das vantagens do \gls{MPS}, é permitir a composição de várias linguagens em uma única solução, isso favorece a separação de preocupações durante a modelagem. Do ponto de vista da primeira \gls{DSL}, a proposta é atender ao público de usuários especialistas no domínio de regras da legislação de cotas. 
 
 Por outro lado, uma \gls{DSL} técnica (para apoio ao desenvolvedores) pode agregar definições necessárias para geração do código fonte, tais como:
 
 \begin{itemize}
     \item Mapeamento de campos relacionados à inscrição do candidato - para definir condições de seleção dos inscritos, de acordo com sua categoria de cota;
     \item Linguagem alvo - para seleção de uma linguagem alvo (ex: Java, PHP) possível de implementação;
     \item Definição da assinatura do método de aprovação de candidatos - que poderá ser importado em uma das linguagens alvo;
     \item Formato de entrada e retorno da lista de candidatos - para indicar ao gerador a estrutura de dados utilizada como entrada ou saída de resultados, podendo ser utilizados por exemplo, formatos como JSON ou CSV;
     \item Composição de campos de inscrição - no sentido de escolha dos campos que serão utilizados no critérios de seleção da base de candidatos inscritos, para cada categoria de cota possível.
 \end{itemize}
 
    No Código Fonte \ref{lst:propostadsldevconfig}, é a apresentada a estrutura proposta para suprir as definições descritas. Na linha 2 seria selecionada a versão de implementação das regras de cotas, e em seguida definidos os dados para geração do método de aprovação (linhas 5, 8, 11, 12). 
    
    \lstinputlisting[language=PHP, 
caption= Configurações para o algortimo de aprovação de candidatos
,label=lst:propostadsldevconfig]{chapters/proposta/codigos_proposta/dsl_tecnica_proposta.m}
 
 Por fim, essa linguagem deveria descrever como os candidatos devem ser selecionados na base, a fim de serem considerados inscritos para uma respectiva categoria de cota, essa situação é proposta por meio da estrutura presente no Código Fonte \ref{lst:dsltecnicacontinuacao}.
 
   \lstinputlisting[language=PHP, 
caption= Composição de campos por tipo de cota
,label=lst:dsltecnicacontinuacao]{chapters/proposta/codigos_proposta/dsl_tecnica_composicao.m}
 
 É importante considerar, que essa linguagem tem carácter de cunho auxiliar, uma vez que um \textit{template} fixo de gerador poderia ser implementado apenas utilizando a primeira \gls{DSL}, que possui as regras base de distribuição. Contudo, após a análise do controle de versão realizada (Capítulo \ref{chap:historicoversoes}), foi observado que grande parte do código escrito ainda estava relacionado com questões técnicas do algoritmo.
 
 Nesse sentido, a elaboração dessa linguagem auxiliar, pode dar celeridade às questões mais técnicas de geração, fornecendo por exemplo: métodos prontos para serem importados em diferentes linguagens de programação, com diferentes tipos de estruturas de dados, ou até a possibilidade de geração de serviço web de aprovação de candidatos.

   
   
   \section{Cronograma}
\label{cronograma}

Nessa Seção é apresentado um cronograma que contém as atividades previstas para conclusão da presente pesquisa:

\begin{table}[!htpb]
    \centering
    \caption{Cronograma}
    \label{tab:cronograma}
    \begin{scriptsize}
    \resizebox{\textwidth}{!}{
    \begin{tabular}{|p{3cm}|c|c|c|c|c|c|c|c|c|c|c|c|}\hline
         &\multicolumn{3}{c|}{1} &
        \multicolumn{3}{c|}{2} &
        \multicolumn{3}{c|}{3} &
        \multicolumn{3}{c|}{4}\\\cline{2-13}
         & 1 & 2 & 3 & 4 & 5 & 6 & 7 & 8 & 9 & 10 & 11 & 12\\\cline{2-13} 
         & 07/19 & 08/19 & 09/19 & 10/19 & 11/19 & 12/19 & 01/20 & 02/20 & 03/20 & 04/20 & 05/20 & 06/20 
         \\\hline    
         
         Revisão da literatura &  & \cellcolor{gray!50} & \cellcolor{gray!50} &\cellcolor{gray!50} & \cellcolor{gray!50}& & & & & & & \\[10pt]\hline
         
         Análise do controle de versão do sistema de cotas &\cellcolor{gray!50} &\cellcolor{gray!50} &\cellcolor{gray!50} & & & & & & & & & \\[10pt]\hline    
         
         Estudo de ferramentas para construção de \gls{DSL} & & & &\cellcolor{gray!50} &\cellcolor{gray!50} &\cellcolor{gray!50} &\cellcolor{gray!50} & & & & & \\[10pt]\hline    
        %  Desenvolvimento do protótipo & & & & & &\cellcolor{gray!50} &\cellcolor{gray!50} &\cellcolor{gray!50} & & & & \\[10pt]\hline    
         
         Implementação das \gls{DSL}s no \gls{MPS} & & & & &\cellcolor{gray!50} &\cellcolor{gray!50} &\cellcolor{gray!50} &\cellcolor{gray!50} &\cellcolor{gray!50} & & & \\[10pt]\hline    
         
         Avaliação de usabilidade da \gls{DSL} & & & & & & & & &\cellcolor{gray!50} &\cellcolor{gray!50} & & \\[10pt]\hline    
         
         Encaminhar dissertação para banca & & & & & & & & & & &\cellcolor{gray!50} & \\[10pt]\hline  
         
         Defesa da dissertação & & & & & & & & & & & &\cellcolor{gray!50} \\[10pt]\hline    
    \end{tabular}}
    \end{scriptsize}
\end{table}

Até o presente momento, estão sendo trabalhadas as atividades de revisão da literatura e a análise do sistema de controle de versão, que originaram este documento. Apesar de, já ter sido iniciado um estudo superficial da ferramenta \gls{MPS}, ainda serão necessários alguns meses para implementação das linguagens propostas.

Após a implementação da \gls{DSL}, pretende-se aplicar questionários para avaliação de usabilidade, com usuários envolvidos na análise da legislação e no desenvolvimento do sistema de cotas.

Por fim, o cronograma prevê o envio da dissertação para os membros da banca e o prazo de defesa. Estima-se que todas as atividades sejam executadas até a data limite de junho de 2020.
   
