\chapter{Proposta}
\label{chap:proposta}

   Nesse capítulo, será descrita a proposta de linguagem a ser desenvolvida na ferramenta \gls{MPS}, que possui como principal objetivo facilitar a definição das regras de distribuição de cotas e dessa forma, permitir que os usuários do sistema de ingresso do IFSC possam alterar essas definições de forma independente à implementação do respectivo algoritmo de classificação.
   
   Como foi descrito no Capítulo \ref{chap:historicoversoes}, desde lançamento da lei de cotas em 2012, o sistema de ingresso passou por, pelo menos, 3 (três) versões diferentes de distribuição de vagas, ocasionando em várias refatorações de código desde então. 
   
   Desse modo, a presente pesquisa tem como proposta: a definição de uma linguagem que permita descrever todos os 3 cenários passados, com apoio de recursos de linguagens existentes no \gls{MPS} tais como: validação de estrutura e de tipos de dados aceitos, preenchimento automático de código de acordo com o contexto, validação de inconsistências, recursos de dicas ao usuário da linguagem e geração de código fonte por meio de transformações. 
   
   Com apoio desses recursos, o usuário especialista no sistema de cotas, poderá fazer o uso controlado da especificação de regras do sistema de cotas, assim como, no caso de surgimento de um novo cenário de distribuição de vagas, esse conjunto de regras alteradas poderiam ser transformadas em código fonte, possibilitando a  sua importação no sistema de ingresso do \gls{IFSC}, diminuindo a necessidade de refatoração em seu código.
   
   Nesse sentido, a Seção \ref{sec:dslproposta:usuario} irá abordar a proposta de uma \gls{DSL} que será utilizada por especialistas de negócio, para definição da árvore de distribuição de cotas, tendo como base as especificações de lei. Enquanto na Seção \ref{sec:dslproposta:dev}, será descrita outra proposta de \gls{DSL}, que utiliza a primeira como dependência, e se aplica a desenvolvedores, porém o foco da modelagem trata de questões técnicas sobre a geração de código do algoritmo de aprovação.
   
   A segunda modelagem de \gls{DSL}, tem como proposta definir: configurações relacionadas à forma de entrada de dados da lista de candidatos, assinatura do método a ser gerado, linguagem alvo da geração, formato da lista de retorno dos aprovados e a composição de campos para cada uma das respectivas categorias de cotas configuradas pelos usuários na primeira \gls{DSL}.
   
   \section{DSL de distribuição de vagas}
\label{sec:dslproposta:usuario}

   
 Tendo como base a análise realizada no Capítulo \ref{chap:historicoversoes}, nas Seções \ref{versao1}, \ref{versao2} e \ref{versao3}, são pontuadas algumas características em comum identificadas entre cada uma dessas versões:
   
   \begin{enumerate}
    \item[a)] A divisão das vagas entre as diferentes categorias de cotas, se deu em formato hierárquico, iniciando no total de vagas o qual foi sendo subdividido em percentuais reservados para suas subcategorias (ex: Estudantes da escola pública, Estudantes com renda baixa, Estudantes PCD, etc); 
   
   \item[b)] Esses percentuais são de conhecimento dos usuários especialistas no sistema de cotas, e podem variar de acordo com a documentação da legislação vigente, ou outras definições que variam conforme a unidade federada que oferta o curso; 
   
   \item[c)] Em muitos casos, não é definido um valor de percentual fixo, de modo que uma categoria recebe o valor calculado restante de vagas da categoria pai. Por exemplo, o percentual de cotistas \gls{PCD} é aplicado, e o que resta das vagas vai para candidatos da mesma categoria que não são \gls{PCD};

   \item[d)] Todas as versões consideravam a ordem de prioridade entre as diferentes categorias de cotas passíveis de inscrição;
   
   \item[e)] Questões de arredondamento das vagas (para cima ou para baixo), devem ser consideradas e podem variar de acordo com a versão de lei implementada.

   \end{enumerate}
   
   Essa \gls{DSL} proposta, objetiva abordar um modelo que permita a configuração de todos os itens listados, tais como: identificador da versão de lei, lista de variáveis para configuração de percentuais, formas de aplicação de arredondamento, uma macro para aplicar a função de resto de vagas de uma categoria, campo descritivo para os diferentes tipos de categorias, estrutura para divisão em subcategorias e lista de ordem de prioridade para sobra de vagas.
   
   No Código Fonte \ref{lst:dsl1versaoconfig}, é apresentado um exemplo de instância da configuração de distribuição de vagas, em que a sintaxe requer uma string para definição da versão, e uma lista de variáveis a serem criadas pelo usuário da linguagem para que sejam definidos os percentuais de reserva de vagas.
   
   \lstinputlisting[language=PHP, 
caption=Configurações de percentuais
,label=lst:dsl1versaoconfig]{chapters/proposta/codigos_proposta/versao_config_proposta.m}


No que diz respeito à estrutura hierárquica, no Código Fonte \ref{lst:dsl1arvorecotas} é possível identificar os campos da árvore de distribuição ("Distribuição de vagas:"), iniciando com o identificador referente à sigla da cota, e suas respectivas propriedades, tais como: descrição da categoria de cota, o percentual de reserva a ser aplicado (variável configurada, valor percentual preenchido manualmente ou a macro "Restante das vagas") e, por fim, se haverá novas subdivisões em outras categorias ("Se divide em:").

   \lstinputlisting[language=PHP, 
caption=Definição da estrutura de distribuição de vagas
,label=lst:dsl1arvorecotas]{chapters/proposta/codigos_proposta/arvore_cotas.m}


O último ponto a ser exemplificado é o caso da configuração da ordem de prioridade ("Ordem prioritária de preenchimento para sobra de vagas:") em caso de sobra de vagas (Código Fonte \ref{lst:dsl1prioridade}), o usuário poderá selecionar as categorias preenchidas anteriormente com a marcação "Quadro de vagas:" de valor "Sim".  

   \lstinputlisting[language=PHP, 
caption=Ordem de preenchimento para sobra de vagas
,label=lst:dsl1prioridade]{chapters/proposta/codigos_proposta/prioridade.m}

É importante ressaltar que, com apoio do \gls{MPS} é possível utilizar recursos de preenchimento automático de código, avisar o usuário da linguagem sobre inconsistências como: declarações duplicadas, formatos inválidos de entradas de dados, categorias de cotas com configurações incompletas ou não preenchidas e ausência de categorias na lista de ordem de prioridade (preenchimento em caso de sobra de vagas). 

Na Seção \ref{sec:dslproposta:dev}, tratamos outros pontos sobre a modelagem para selecionar candidatos em banco de dados, ou em outro tipo de estrutura de dados. Nesse sentido, a proposta é separar questões mais técnicas em uma linguagem à parte, de modo que a \gls{DSL} descrita anteriormente, seja utilizada como dependência, e seja feito o mapeamento com informações para geração de código fonte em uma linguagem alvo.
   
 Tendo como base a análise realizada no Capítulo \ref{chap:historicoversoes} nas seções \ref{versao1}, \ref{versao2} e \ref{versao3}, serão pontuadas algumas características em comum identificadas entre cada uma dessas versões:
   
   \begin{itemize}
    \item A divisão das vagas entre as diferentes categorias de cotas, se deu em formato hierárquico, iniciando no total de vagas o qual foi sendo subdividido em percentuais reservados para suas subcategorias (ex: Estudantes da escola pública, Estudantes com renda baixa, Estudantes PCD, etc); 
   
   \item Esses percentuais, são de conhecimento dos usuários especialistas no sistema de cotas, e podem variar de acordo com a documentação da legislação vigente, ou outras definições que variam conforme a unidade federada que oferta o curso; 
   
   \item Em muitas vezes, é aplicando um percentual a uma categoria, e a cota seguinte fica com o restante das vagas. Por exemplo, estudantes da categoria Escola Pública (EP), que possuem Renda Inferior (RI) ou igual a 1.5 salários mínimos, que são \gls{PPI}, possuem direito à 15.7\% das vagas em relação aos que não são \gls{PPI};
   
   \item Apenas alguns nós, geralmente as folhas, da árvore de distribuição de cotas são consideradas como passíveis de inscrição por candidatos, são neles que estarão os valores finais de reserva para cada categoria de cota;
   
   \item Todas as versões consideravam a ordem de prioridade entre as diferentes categorias de cotas passíveis de inscrição;
   
   \item Questões de arredondamento das vagas (para cima ou para baixo), devem ser consideradas e podem variar de acordo com a versão de lei implementada.

   \end{itemize}
   
   
   
   
   
   

   \section{DSL técnica para geração algoritmo de aprovação}
\label{sec:dslproposta:dev}
   
