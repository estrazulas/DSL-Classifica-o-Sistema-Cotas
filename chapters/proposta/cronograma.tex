\section{Cronograma}
\label{cronograma}

Nessa seção será apresentado um cronograma que contém as atividades previstas para conclusão da presente pesquisa:

\begin{table}[!htpb]
    \centering
    \caption{Cronograma}
    \label{tab:cronograma}
    \begin{scriptsize}
    \resizebox{\textwidth}{!}{
    \begin{tabular}{|p{3cm}|c|c|c|c|c|c|c|c|c|c|c|c|}\hline
         &\multicolumn{3}{c|}{1} &
        \multicolumn{3}{c|}{2} &
        \multicolumn{3}{c|}{3} &
        \multicolumn{3}{c|}{4}\\\cline{2-13}
         & 1 & 2 & 3 & 4 & 5 & 6 & 7 & 8 & 9 & 10 & 11 & 12\\\cline{2-13} 
         & 07/19 & 08/19 & 09/19 & 10/19 & 11/19 & 12/19 & 01/20 & 02/20 & 03/20 & 04/20 & 05/20 & 06/20 
         \\\hline    
         
         Revisão da literatura &  & \cellcolor{gray!50} & \cellcolor{gray!50} &\cellcolor{gray!50} & \cellcolor{gray!50}& & & & & & & \\[10pt]\hline
         
         Análise do controle de versão do sistema de cotas &\cellcolor{gray!50} &\cellcolor{gray!50} &\cellcolor{gray!50} & & & & & & & & & \\[10pt]\hline    
         
         Estudo de ferramentas para construção de \gls{DSL} & & & &\cellcolor{gray!50} &\cellcolor{gray!50} &\cellcolor{gray!50} &\cellcolor{gray!50} & & & & & \\[10pt]\hline    
        %  Desenvolvimento do protótipo & & & & & &\cellcolor{gray!50} &\cellcolor{gray!50} &\cellcolor{gray!50} & & & & \\[10pt]\hline    
         
         Implementação das \gls{DSL}s no \gls{MPS} & & & & &\cellcolor{gray!50} &\cellcolor{gray!50} &\cellcolor{gray!50} &\cellcolor{gray!50} &\cellcolor{gray!50} & & & \\[10pt]\hline    
         
         Avaliação de usabilidade da \gls{DSL} & & & & & & & & &\cellcolor{gray!50} &\cellcolor{gray!50} & & \\[10pt]\hline    
         
         Encaminhar dissertação para banca & & & & & & & & & & &\cellcolor{gray!50} & \\[10pt]\hline  
         
         Defesa da dissertação & & & & & & & & & & & &\cellcolor{gray!50} \\[10pt]\hline    
    \end{tabular}}
    \end{scriptsize}
\end{table}

Até o presente momento, estão sendo trabalhadas as atividades de revisão da literatura e a análise do sistema de controle de versão, que originaram este documento. Apesar de, já ter sido iniciado um estudo superficial da ferramenta \gls{MPS}, ainda serão necessários alguns meses para implementação das linguagens propostas.

Após a implementação da \gls{DSL}, pretende-se aplicar questionários para avaliação de usabilidade, com usuários envolvidos na análise da legislação e no desenvolvimento do sistema de cotas.

Por fim, o cronograma prevê o envio da dissertação para os membros da banca e o prazo de defesa. Estima-se que todas as atividades sejam executadas até a data limite de junho de 2020.