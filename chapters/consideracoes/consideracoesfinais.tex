\chapter{Conclusões}
\label{chap:consideracoes}

\section{Trabalhos Correlatos}
\label{subsection:trabalhoscorrelatos}
 Segundo \citeonline{tahawalid2009}, existem duas maneiras diferentes pelas quais um domínio pode ser definido. A primeira é quando o domínio está bem definido matematicamente, a segunda quando o domínio é definido por atividades puramente humanas. As DSLs de domínio humano, definem uma linguagem especial ou jargões para comunicar ideias relacionadas ao seu domínio.  
 
 Nesse contexto, os trabalhos correlatos encontrados mostraram formas de abstrair conceitos e jargões envolvendo questões legais, contratuais e regras sobre domínios financeiros. Os domínios dessas DSLs foram definidos com objetivo de apoiar na compreensão de atividades humanas e não de tratar de problemas computacionais ou matemáticos.
 
 Para tanto, as DSLs descritas nas próximas Seções foram encontradas por meio da busca pelas palavras-chave (domain specific languages, DSL, law e legal)  nos repositórios, \textit{Researchgate}, \textit{Google Scholar} e \textit{IEEEXplore}.
 
 
\subsection{LegalLanguage}
\label{legallanguage}

A DSL \texttt{LegalLanguage} é resultado da análise de ontologias que tratam de estrutura de documentos legais, sua construção busca estabelecer modelos conceituais com propósito de melhoria de comunicação e aprendizado sobre definições encontradas em textos normativos.

\begin{figure}[ht!]
\centering

\caption{\textmd{Modelo da DSL LegalLanguage}}
\label{fig:legallanguagemodel}
\fcolorbox{gray}{white}{\includegraphics[width=0.90\textwidth]{chapters/trabalhoscorrelatos/imagens/legallanguagemodel.png}}

\par\medskip\textbf{Fonte:} \citeonline{legallanguage}. \par\medskip

\end{figure}



Segundo \citeonline{legallanguage}, o seu modelo é composto de \textit{Laws}, que podem ser classificadas como: Leis internacionais, regulamentos públicos, regulamentos privados, constituições e leis ordinárias. As leis são compostas por uma série artigos ordenados sequencialmente, os quais são organizados em divisões. É possível criar relacionamentos hierárquicos entre artigos de diferentes elementos de lei, bem como indicar revogações quando novas leis surgem (Figura \ref{fig:legallanguagemodel}). 

Segundo \citeonline[p.45, tradução nossa]{legallanguage}: "O desenvolvimento dessa linguagem tem como objetivo apoiar parlamentares que executam atividades inseridas em processos legislativos de redação de leis". Portanto, os autores à criaram utilizando a ferramenta \texttt{Xtext}, um exemplo de definição e uso do elemento \texttt{Law} pode ser observado nas Figuras \ref{fig:xtextlegal} e \ref{fig:legallanguageexample}.

\begin{figure}[ht!]
\centering

\caption{\textmd{Definição do elemento Law na LegalLanguage}}
\label{fig:xtextlegal}
\fcolorbox{gray}{white}{\includegraphics[width=0.85\textwidth]{chapters/trabalhoscorrelatos/imagens/xtextlegal.png}}

\par\medskip\textbf{Fonte:} \cite{legallanguage}. \par\medskip

\end{figure}




\begin{figure}[ht!]
\centering

\caption{\textmd{Exemplo ilustrativo da LegalLanguage}}
\label{fig:legallanguageexample}
\fcolorbox{gray}{white}{\includegraphics[width=0.90\textwidth]{chapters/trabalhoscorrelatos/imagens/legallanguageexample.png}}

\par\medskip\textbf{Fonte:} \cite{legallanguage}. \par\medskip

\end{figure}





\newpage
\subsection{Ergo uma DSL para Contratos Legais Inteligentes}
\label{ergo}

Ergo é uma linguagem específica de domínio que faz parte do projeto Accord, o qual define a base jurídica e técnica para contratos legais inteligentes, com objetivo de atender à problemas relacionados a falta de padronização em contratos legais. 

Segundo \citeonline{accordproject}, a linguagem tem o foco de ajudar desenvolvedores da área de tecnologia jurídica na escrita de contratos legais que possam ser computados. Um contrato legal inteligente é um contrato legível por seres humanos e por máquinas, por exemplo, uma clausula de cobrança de pagamento pode estar presente em um contrato, de modo que se possa utilizar o texto descrito em linguagem natural, para extração dos dados de cobrança e posterior aplicação de cálculos de multas e geração de eventos de notificações. 


É uma linguagem fortemente tipada e independente de plataforma, seu código pode ser compilado nas plataformas \texttt{JavaScrit} e \texttt{Java}. Ela prove uma linguagem de expressões, pela qual é possível descrever funções e clausulas de modo que possa ser definida a lógica de contratos inteligentes. 

Um exemplo de linguagem natural pode ser observado na Figura \ref{fig:ergotexto}, no qual os dados sobre a clausula de cobrança são mapeados, para posteriormente serem modelados e terem a lógica de cobrança seja definida (Figuras \ref{fig:ergomodelo} e \ref{fig:ergologica}). 

\begin{figure}[ht!]
\centering

\caption{\textmd{Exemplo ilustrativo da Ergo DSL}}
\label{fig:ergotexto}
\fcolorbox{gray}{white}{\includegraphics[width=\textwidth]{chapters/trabalhoscorrelatos/imagens/ergotexto.PNG}}

\par\medskip\textbf{Fonte:} \citeonline{accordproject}. \par\medskip

\end{figure}



\begin{figure}[ht!]
\centering

\caption{\textmd{Modelagem de clausula na ERGO DSL}}
\label{fig:ergomodelo}
\fcolorbox{gray}{white}{\includegraphics[width=0.90\textwidth]{chapters/trabalhoscorrelatos/imagens/ergomodelo.PNG}}

\par\medskip\textbf{Fonte:} \citeonline{accordproject}. \par\medskip

\end{figure}



\begin{figure}[ht!]
\centering

\caption{\textmd{Definição da lógica de contratos}}
\label{fig:ergologica}
\fcolorbox{gray}{white}{\includegraphics[width=0.90\textwidth]{chapters/trabalhoscorrelatos/imagens/ergologica.PNG}}

\par\medskip\textbf{Fonte:} \citeonline{accordproject}. \par\medskip

\end{figure}



\newpage

\subsection{DSL Capgemini Pension}
\label{capgeminipension}

Segundo \citeonline{gregfuller2013}, foi criada com o objetivo de atender ao domínio de planos de previdência e fornecimento de planos de seguros de pensão, no qual cada plano pode conter centenas de regras específicas, tratadas conforme o histórico de até 40 anos de dados funcionais para milhares de segurados. 

 Para \citeonline{kolkhenk2008}, essa linguagem advém da necessidade de inovação em produtos da área e é baseada em interesses governamentais, no sentido de que há a necessidade de adequação à novas leis de pensão, dar transparência sobre as regras e fornecer garantia de qualidade. 



Criada com a ferramenta \texttt{Intentional Software} para construção de DSLs projecionais, ela permite que as definições das regras dos planos sejam realizadas em formato gráfico no formato de tabelas (Figura \ref{fig:dslcapgeminitables}) e permite fazer testes unitários para fazer validações das regras em tempo real (Figura \ref{fig:dslcapgeminiunittests}). 

\begin{figure}[ht!]
\centering

\caption{\textmd{Editor projecional das regras de pensão}}
\label{fig:dslcapgeminitables}
\fcolorbox{gray}{white}{\includegraphics[width=0.95\textwidth]{chapters/trabalhoscorrelatos/imagens/dslcapgeminitables.PNG}}

\par\medskip\textbf{Fonte:} \citeonline{kolkhenk2008}. \par\medskip

\end{figure}



\begin{figure}[ht!]
\centering

\caption{\textmd{Testes unitários das regras}}
\label{fig:dslcapgeminiunittests}
\fcolorbox{gray}{white}{\includegraphics[width=\textwidth]{chapters/trabalhoscorrelatos/imagens/dslcapgeminiunittests.png}}

\par\medskip\textbf{Fonte:} \cite{kolkhenk2008}. \par\medskip

\end{figure}



O ambiente da linguagem substitui uma série de planilhas, documentos de texto e outras fontes de controle utilizadas por múltiplos especialistas de domínio no contexto de gerenciamento dos planos de pensão. Dessa forma, a \textit{Pension Workbench} traz um ambiente colaborativo controlado, com sistema integrado de controle de versão, sendo passível criar testes unitários e gerar a \textit{engine} responsável pelos cálculos para a administração dos planos e fundos de pensão \cite{gregfuller2013}.



\section{Escopo negativo da pesquisa}
\label{escoponegativo}

Essa pesquisa é proveniente dos decorrentes problemas de entendimento, comunicação e manutenção dos sistemas do \gls{IFSC} que utilizam regras relacionadas ao sistema de cotas da rede de ensino pública federal. Nesse contexto, procurou-se aplicar conceitos sobre linguagens específicas de domínio como meio de apoio durante as etapas de descrição e implementação das regras por parte usuários especialistas e desenvolvedores. 

Portanto, limitou-se ao contexto da realidade do \gls{IFSC}, no sentido de que a análise foi baseada em documentações internas, no sistema de controle de versão institucional e com usuários finais em sua maioria atuantes no setor \gls{DEING}. Apesar de ser baseada em uma legislação utilizada por várias instituições públicas federais, os resultados apresentados podem ter entendimento distintos por outras organizações ou setores, uma vez que os textos presentes na lei podem estar sujeitos a outras interpretações. Contudo, essas instituições podem se beneficiar da estrutura base da DSL desenvolvida adequando as definições conforme sua realidade e características regionais.

Ademais, o modelo da linguagem desenvolvida foi estabelecido a partir de elementos identificados no histórico de modificações dos documentos de lei de 2012 até o presente momento, tentou-se obter um modelo que pudesse contemplar novas modificações em lei, no entanto, as legislações futuras podem estar sujeitas a total reformulação ou até a sua extinção, o que não pode ser previsto no contexto do presente estudo, ficando a cargo dos designers da linguagem fazer as devidas atualizações de modelo. Cabe destacar que as adequações no modelo são facilitadas pelos recursos disponíveis na ferramenta MPS, o qual permite que as DSLs sejam estendidas e reutilizadas em outros contextos de aplicação.

Por uma questão de disponibilidade e de tempo hábil para aprofundamento do estudo foram selecionados 20 usuários para a realização do exercício com a DSL, alguns dos grupos de perfil de usuários tiveram menor representatividade em relação aos demais. Portanto, a análise limitou-se aos usuários que concordaram em participar da pesquisa, na qual tentou-se obter uma variedade significativa e relevante para o estudo com usuários especialistas, desenvolvedores e leigos. No entanto, entende-se que este número é suficiente para validar os objetivos da presente pesquisa, considerando as afirmações de \citeonline{nielsen2012many}, que sugere que testes de usabilidade com apenas 5 (cinco) usuários permitem encontrar quase tantos problemas de usabilidade do que quando se utilizam estudos com mais participantes. Contudo, o referido autor defende que no caso de usuários de diferentes perfis, recomenda-se a aplicação de 3 (três) a 4 (quatro) usuários por grupo, porque as experiências se sobrepõem e ampliam a cobertura dos testes de usabilidade. 

Por fim, sobre a construção da API como prova de conceito sobre a modelagem da DSL Cotas, por uma questão de limitação de acesso aos dados históricos de candidatos, foi realizada a validação com dados históricos de candidatos disponíveis no banco de dados do sistema de ingresso do \gls{IFSC}. Essa validação não foi aplicada em outras bases de dados externas, o que poderia ampliar o estudo sobre outras realidades de resultados em processos seletivos fora da instituição.

\section{Principais contribuições}
\label{principaiscontribuicoes}

A presente pesquisa contribuiu para o aprofundamento do conhecimento sobre estudos que utilizam \gls{DSL}s no contexto de modelagem para aplicação de regras estabelecidas em legislação. Do mesmo modo, avaliam-se os benefícios e os desafios da aplicação dessa proposta de engenharia de software para instituições federais de ensino que aplicam em seus processos seletivos as regras de distribuição de cotas. 

Outra contribuição advém da análise dos resultados coletados com os 4 (quatro) grupos de usuários, no sentido de apresentar os diferentes relatos de dificuldades e preocupações sobre o uso de DSLs. Esse levantamento sugere que a validade de aplicação de uma nova linguagem pode variar de acordo com o nível de conhecimento na área de negócio e conforme o contexto de experiências profissionais e acadêmicas dos usuários.

Destaca-se, igualmente, a contribuição no âmbito pessoal e profissional do pesquisador em relação aos conhecimentos adquiridos por meio da pesquisa. Esses conhecimentos dizem respeito ao estudo de problemas causados durante a evolução do sistema de ingresso institucional, à identificação de padrões de elementos e regras presentes nas diferentes versões da legislação, o uso de ferramentas projecionais para construção de linguagens específicas de domínio, assim como a geração de código de processamento de candidatos a partir da sintaxe extraída da DSL.

Por fim, infere-se que além da relevância teórico-prática para a implementação de linguagem de domínio específico, o estudo aqui desenvolvido possui uma relevância social ao possibilitar maior comunicação entre a área de negócio e de desenvolvimento. E, assim, garantir que as políticas institucionais de equiparação de oportunidades para o ingresso de cotistas na rede federal de ensino sejam implementadas com mais produtividade e de modo colaborativo entre os envolvidos.   

\section{Trabalhos Futuros}
\label{trabalhosfuturos}

Como proposta de trabalhos futuros sugere-se:

\begin{enumerate}
    \item [a)] Verificar a adesão da DSL Cotas por usuários especialistas e desenvolvedores de outras instituições de ensino pública federais;
    \item [b)] Disponibilizar publicamente os serviços da API DSL Cotas para implantação em redes de ensino federadas, com intuito de verificar a aplicação em outras bases de dados por meio da colaboração e testes com outros setores de desenvolvimento de sistemas da rede; 
    \item [c)] Criar um repositório de controle de versão centralizado para publicação das definições de regras de cotas presentes nas diferentes instituições e estados. De modo que se possa verificar a transparência nas regras utilizadas e garantir a rastreabilidade de alterações realizadas conforme cada edital de processo seletivo que requer atendimento da legislação de cotas;
    \item [d)] Investigar outras formas de visualização e geração de código a partir da DSL Cotas, tais como: geração de editais, geração do quadro de vagas, geração de resultados e geração de elementos de interface gráfica para sistemas e aplicativos de inscrição em processos seletivos de modo que possam ser apresentadas as definições de categorias durante o processo de inscrição até o momento final de classificação;
    \item [e)] Avaliar e modelar alterações na DSL Cotas para aplicação em outros tipos de processos seletivos que envolvam classificação por cotas fora do âmbito da Lei nº 12.711.
\end{enumerate}

\section{Considerações finais}
\label{considfinal}

O avanço na complexidade das regras de negócio traz consigo uma série de discussões e estudos que objetivam abstrair e simplificar a sua implementação nos sistemas de informação, de modo a reduzir os impactos durante o seu desenvolvimento, podendo melhorar o desempenho das organizações.

As ferramentas modernas de \gls{DSL} permitem a especificação de gramáticas e estruturas de linguagem, sem que haja a necessidade de manualmente criar um \textit{parser}. Elas permitem a validação e a criação de estruturas e tipos de dados, assim como fornecem uma série de recursos para apoio ao usuário da linguagem, podendo agilizar o processo de definição de regras e a respectiva geração de código fonte.

Nesse contexto, a presente pesquisa buscou compreender, por meio da elaboração de uma linguagem específica de domínio, a viabilidade de melhoria na comunicação entre usuários de negócio e desenvolvedores, no que concerne à especificação de requisitos e a implementação de regras relacionadas ao sistema de cotas da rede de ensino pública federal.

A pesquisa apresentada buscou contextualizar uma série de demandas de alteração no sistema de ingresso do \gls{IFSC}, que trazem dependência entre os especialistas de domínio e os desenvolvedores, no sentido de que as regras precisam ser traduzidas em várias linhas de código, para que sejam criadas as funcionalidades que implementam a classificação de candidatos conforme os requisitos de lei.

Portanto, para a criação da DSL Cotas foram fundamentados os principais conceitos de linguagens específicas de domínio, que possibilitaram definir uma sintaxe de definição de regras mais simples para o usuário tratar de questões de negócio, sem que esses precisem se preocupar com conhecimentos mais complexos de programação, mas ainda assim, colaborando com o processo de modelagem e construção da solução.

Adicionalmente ao desenvolvimento da DSL Cotas, foi construída uma \gls{API} que importa as definições da linguagem para disponibilizar o serviço de classificação e aprovação de candidatos, servindo também como prova de conceito e validade da modelagem estabelecida por esse estudo.

Os sujeitos que fizeram parte dessa investigação foram 20 usuários com perfis diferentes, que foram divididos em 4 (quatro) grupos nomeados de DEV-ESP, DEV-NESP, NDEV-EPS e NDEV-NESP. Esses usuários foram convidados a responder a um exercício de utilização da DSL Cotas e um questionário de avaliação qualitativa da experiência de uso. 
 
Dessas ações - o levantamento de dados do histórico de versionamento, o desenvolvimento da DSL, a aplicação do exercício e o questionário de avaliação - resultou a investigação para responder ao problema de pesquisa. 

Em relação ao levantamento do sistema de controle de versão, foi possível identificar as diferenças de implementação entre as alterações de lei já realizadas no \gls{IFSC}. A respeito do objetivo específico de analisar essas características foi possível atender ao terceiro objetivo, que diz respeito à modelagem de regras por meio de uma linguagem específica de domínio. Com o intuito de verificar o quarto objetivo específico, que trata da avaliação de uso da DSL pelos usuários convidados, observou-se que o grupo que teve mais facilidade de entendimento e utilização dos recursos da DSL Cotas foi o grupo DEV-ESP, no entanto, o grupo NDEV-ESP foi o que conseguiu configurar distribuições mais recentes da legislação, sem que fosse solicitado. Os demais grupos DEV-NESP e NDEV-NESP apontaram um grau de dificuldade maior em relação ao seu uso em função de lacunas de conhecimento sobre domínio ou por dificuldades técnicas.

Assim, por meio da pesquisa realizada compreende-se que a abordagem de aplicação da DSL Cotas no contexto de definição de regras da legislação, mostrou-se viável para contribuir na melhoria de comunicação entre usuários de negócio e desenvolvedores, inferindo-se desse modo, que há a possibilidade no aumento de produtividade da especificação de requisitos e a respectiva implementação de regras concernentes ao sistema de cotas da rede de ensino pública federal.


