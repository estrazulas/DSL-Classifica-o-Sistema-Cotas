\chapter{Considerações Finais}
\label{chap:consideracoes}



O avanço na complexidade das regras de negócio, traz consigo uma série de discussões e estudos que objetivam abstrair e simplificar a sua implementação nos sistemas de informação, de modo a reduzir os impactos durante o seu desenvolvimento, podendo melhorar o desempenho das organizações.   

Nesse contexto, a pesquisa proposta busca atacar as dificuldades encontradas durante a recorrente demanda por refatoração de código fonte do sistema de ingresso do \gls{IFSC}, em função de mudanças nas regras estabelecidas em leis, decretos e instruções normativas de órgãos de controle.

A pesquisa contextualiza uma série de demandas de alteração no sistema de ingresso do \gls{IFSC}, que trazem forte dependência entre os especialistas de domínio e os desenvolvedores, no sentido de que as regras precisam ser traduzidas em várias linhas de código, para que sejam criadas as funcionalidades que implementam classificação de candidatos conforme os requisitos de lei.

Portanto, são fundamentados os principais conceitos de linguagens específicas de domínio, que possibilitam definir uma sintaxe mais simples para o usuário, para tratar apenas questões de negócio, sem que esses precisem se preocupar com conhecimentos mais complexos de programação, mas ainda assim colaborando com o processo de modelagem e construção da solução.

As ferramentas modernas de \gls{DSL}, permitem a especificação de gramáticas e estruturas de linguagem, sem que haja a necessidade de manualmente criar um \textit{parser}. Elas permitem a validação e criação de estruturas e tipos de dados, assim como fornecem uma série de recursos para apoio ao usuário da linguagem, podendo agilizar o processo de definição de regras, e a respectiva geração de código fonte.

Por meio dos conceitos abordados e da justificativa de escolha da ferramenta \gls{MPS}, a presente pesquisa propõe duas \gls{DSL}s, uma com foco no usuário especialista de negócio e outra que objetiva a configuração dos parâmetros de geração do código fonte (foco no desenvolvedor). Ambas são utilizadas para atingir o mesmo objetivo, porém tratam de preocupações diferentes: conceitos necessários para definição de regras de negócio e conceitos de programação para desenvolvedores que irão implementar o gerador do algoritmo resultante.

Será proposta uma pesquisa qualitativa com os usuários de negócio, na qual a usabilidade da linguagem será avaliada por meio de experimento. Por fim, foi definido um cronograma de 12 meses, descrevendo as atividades necessárias para conclusão dos objetivos dessa 
pesquisa.

\section{Trabalhos Correlatos}
\label{subsection:trabalhoscorrelatos}


\subsection{Ergo uma \textit{DSL} para acordos Legais}
\label{ergo}


\subsection{DSLFIN uma \textit{DSL} para sistemas financeiros}
\label{dslfin}

\section{Principais contribuições}
\label{principaiscontribuicoes}

\section{Escopo negativo da pesquisa}
\label{escoponegativo}

\section{Trabalhos Futuros}
\label{trabalhosfuturos}