\chapter{Conclusões}
\label{chap:consideracoes}

\section{Trabalhos Correlatos}
\label{subsection:trabalhoscorrelatos}


\subsection{Ergo uma \textit{DSL} para acordos Legais}
\label{ergo}


\subsection{DSLFIN uma \textit{DSL} para sistemas financeiros}
\label{dslfin}

\section{Escopo negativo da pesquisa}
\label{escoponegativo}

Essa pesquisa é proveniente dos decorrentes problemas de entendimento, comunicação e manutenção dos sistemas do \gls{IFSC} que utilizam regras relacionadas ao sistema de cotas da rede de ensino pública federal. Nesse contexto, procurou-se aplicar conceitos sobre linguagens específicas de domínio como meio de apoio durante as etapas de descrição e implementação das regras por parte usuários especialistas e desenvolvedores. 

Portanto, limitou-se ao contexto da realidade do \gls{IFSC}, no sentido de que a análise foi baseada em documentações internas, no sistema de controle de versão institucional e com usuários finais em sua maioria atuantes no setor \gls{DEING}. Apesar de ser baseada em uma legislação utilizada por várias instituições públicas federais, os resultados apresentados podem ter entendimento distintos por outras organizações ou setores, uma vez que os textos presentes na lei podem estar sujeitos a outras interpretações. 

Por uma questão de disponibilidade e tempo hábil para aprofundamento do estudo, foram selecionados apenas 20 usuários para realização do exercício com a DSL, alguns dos grupos de perfil de usuários tiveram menor representatividade em relação aos demais. Portanto, a análise limitou-se aos usuários que concordaram em participar da pesquisa, na qual tentou-se obter uma variedade significativa e relevante para o estudo com usuários especialistas, desenvolvedores e leigos.

Por fim, sobre a construção da API como prova de conceito sobre a modelagem da DSL Cotas, por uma questão de limitação de acesso aos dados históricos de candidatos, foi realizada a validação com dados históricos de candidatos disponíveis no banco de dados do sistema de ingresso do \gls{IFSC}, essa validação não foi aplicada em outras bases de dados externas, o que poderia ampliar o estudo sobre outras realidades de resultados em processos seletivos fora da instituição.

\section{Principais contribuições}
\label{principaiscontribuicoes}

A presente pesquisa contribuiu para o aprofundamento do conhecimento sobre estudos que utilizam \gls{DSL}s no contexto de modelagem para aplicação de regras estabelecidas em legislação. Do mesmo modo, avaliam-se os benefícios e os desafios da aplicação dessa proposta de engenharia de software para instituições federais de ensino que aplicam em seus processos seletivos as regras de distribuição de cotas. 

Outra contribuição advém da análise dos resultados coletados com os 4 (quatro) grupos de usuários, no sentido de apresentar os diferentes relatos de dificuldades e preocupações sobre o uso de DSLs. Esse levantamento sugere que a validade de aplicação de uma nova linguagem pode variar de acordo com o nível de conhecimento na área de negócio e conforme o contexto de experiências profissionais e acadêmicas dos usuários.

Destaca-se, igualmente, a contribuição no âmbito pessoal e profissional do pesquisador em relação aos conhecimentos adquiridos por meio da pesquisa. Esses conhecimentos dizem respeito ao estudo de problemas causados durante a evolução do sistema de ingresso institucional, à identificação de padrões de elementos e regras presentes nas diferentes versões da legislação, o uso de ferramentas projecionais para construção de linguagens específicas de domínio, assim como a geração de código de processamento de candidatos a partir da sintaxe extraída da DSL.

Por fim, infere-se que além da relevância teórico-prática para a implementação de linguagem de domínio específico, o estudo aqui desenvolvido possui uma relevância social ao possibilitar maior comunicação entre a área de negócio e de desenvolvimento. E, assim, garantir que as políticas institucionais de equiparação de oportunidades para o ingresso de cotistas na rede federal de ensino sejam implementadas com mais produtividade e de modo colaborativo entre os envolvidos.   

\section{Trabalhos Futuros}
\label{trabalhosfuturos}

Como proposta de trabalhos futuros sugere-se:

\begin{enumerate}
    \item [a)] Verificar a adesão da DSL Cotas por usuários especialistas e desenvolvedores de outras instituições de ensino pública federais;
    \item [b)] Disponibilizar publicamente os serviços da API DSL Cotas para implantação em redes de ensino federadas, com intuito de verificar a aplicação em outras bases de dados por meio da colaboração e testes com outros setores de desenvolvimento de sistemas da rede; 
    \item [c)] Criar um repositório de controle de versão centralizado para publicação das definições de regras de cotas presentes nas diferentes instituições e estados. De modo que se possa verificar a transparência nas regras utilizadas e garantir a rastreabilidade de alterações realizadas conforme cada edital de processo seletivo que requer atendimento da legislação de cotas;
    \item [d)] Investigar outras formas de visualização e geração de código a partir da DSL Cotas, tais como: geração de editais, geração do quadro de vagas, geração de resultados e geração de elementos de interface gráfica para sistemas e aplicativos de inscrição em processos seletivos de modo que possam ser apresentadas as definições de categorias durante o processo de inscrição até o momento final de classificação;
    \item [e)] Avaliar e modelar alterações na DSL Cotas para aplicação em outros tipos de processos seletivos que envolvam classificação por cotas fora do âmbito da Lei nº 12.711.
\end{enumerate}

\section{Considerações finais}
\label{considfinal}

O avanço na complexidade das regras de negócio, traz consigo uma série de discussões e estudos que objetivam abstrair e simplificar a sua implementação nos sistemas de informação, de modo a reduzir os impactos durante o seu desenvolvimento, podendo melhorar o desempenho das organizações.   

Nesse contexto, a pesquisa proposta busca atacar as dificuldades encontradas durante a recorrente demanda por refatoração de código fonte do sistema de ingresso do \gls{IFSC}, em função de mudanças nas regras estabelecidas em leis, decretos e instruções normativas de órgãos de controle.

A pesquisa contextualiza uma série de demandas de alteração no sistema de ingresso do \gls{IFSC}, que trazem forte dependência entre os especialistas de domínio e os desenvolvedores, no sentido de que as regras precisam ser traduzidas em várias linhas de código, para que sejam criadas as funcionalidades que implementam classificação de candidatos conforme os requisitos de lei.

Portanto, são fundamentados os principais conceitos de linguagens específicas de domínio, que possibilitam definir uma sintaxe mais simples para o usuário, para tratar apenas questões de negócio, sem que esses precisem se preocupar com conhecimentos mais complexos de programação, mas ainda assim colaborando com o processo de modelagem e construção da solução.

As ferramentas modernas de \gls{DSL}, permitem a especificação de gramáticas e estruturas de linguagem, sem que haja a necessidade de manualmente criar um \textit{parser}. Elas permitem a validação e criação de estruturas e tipos de dados, assim como fornecem uma série de recursos para apoio ao usuário da linguagem, podendo agilizar o processo de definição de regras, e a respectiva geração de código fonte.

Por meio dos conceitos abordados e da justificativa de escolha da ferramenta \gls{MPS}, a presente pesquisa propõe duas \gls{DSL}s, uma com foco no usuário especialista de negócio e outra que objetiva a configuração dos parâmetros de geração do código fonte (foco no desenvolvedor). Ambas são utilizadas para atingir o mesmo objetivo, porém tratam de preocupações diferentes: conceitos necessários para definição de regras de negócio e conceitos de programação para desenvolvedores que irão implementar o gerador do algoritmo resultante.

Será proposta uma pesquisa qualitativa com os usuários de negócio, na qual a usabilidade da linguagem será avaliada por meio de experimento. Por fim, foi definido um cronograma de 12 meses, descrevendo as atividades necessárias para conclusão dos objetivos dessa 
pesquisa.
